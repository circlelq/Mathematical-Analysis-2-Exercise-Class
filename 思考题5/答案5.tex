
\subsection{1}

设 $f_{x}(x, y)$ 在 $\left(x_{0}, y_{0}\right)$ 存在, $f_{y}(x, y)$ 在 $\left(x_{0}, y_{0}\right)$ 连续。

求证: $f(x, y)$ 在 $\left(x_{0}, y_{0}\right)$ 可微。

\begin{proof}
	

\begin{align}
f\left(x_{0}+\Delta x, y_{0}+\Delta y\right)-f\left(x_{0}, y_{0}\right) =&f\left(x_{0}+\Delta x, y_{0}+\Delta y\right)-f\left(x_{0}+\Delta x, y_{0}\right)\\
&+f\left(x_{0}+\Delta x, y_{0}\right)-f\left(x_{0}, y_{0}\right) \\
=&f_y\left(x_{0}+\Delta x, y_{0}+\theta \Delta y\right) \Delta y+f_x\left(x_{0}, y_{0}\right) \Delta x+o(\Delta x) \\
=&f_y\left(x_{0}, y_{0}\right) \Delta y+\alpha \Delta y+f_x\left(x_{0}, y_{0}\right) \Delta x+o(\Delta x),
\end{align}
其中$\alpha=f_y\left(x_{0}+\Delta x, y_{0}+\theta \Delta y\right)-f_y\left(x_{0}, y_{0}\right), \quad 0<\theta<1$ 

当$\Delta x, \Delta y \rightarrow 0 \text { 时, } \alpha \rightarrow 0 $,

$\therefore  \lim \limits_{\Delta x \rightarrow 0, \Delta y \rightarrow 0} \alpha \Delta y / \sqrt{\Delta x^{2}+\Delta y^{2}}=0 $.

$\therefore \alpha \Delta y=o\left(\sqrt{\Delta x^{2}+\Delta y^{2}}\right) $.

$\therefore f\left(x_{0}+\Delta x, y_{0}+\Delta y\right)-f\left(x_{0}, y_{0}\right)=f_y\left(x_{0}, y_{0}\right) \Delta y+f_x\left(x_{0}, y_{0}\right) \Delta x+o\left(\sqrt{\Delta x^{2}+\Delta y^{2}}\right)$.

\end{proof}


附: 二元函数 $f(x, y)$ 在 $\left(x_{0}, y_{0}\right)$ 连续,则在 $\left(x_{0}, y_{0}\right)$ 临域内由
\begin{align}
\left|f\left(x_{0}+\Delta x, y_{0}+\Delta y\right)-f\left(x_{0}+\Delta x, y_{0}\right)\right| \leq &\left|f\left(x_{0}+\Delta x, y_{0}+\Delta y\right)-f\left(x_{0}, y_{0}\right)\right| \\
&+\left|f\left(x_{0}+\Delta x, y_{0}\right)-f\left(x_{0}, y_{0}\right)\right|
\end{align}
可得 $f\left(x_{0}+\Delta x, y\right)$ 在 $\left(x_{0}+\Delta x, y_{0}\right)$ 关于 ${y}$ 连续.

\subsection{2}

若函数 $f(x, y, z)$ 对任意正实数 ${t}$ 满足关系 $f(t x, t y, t z)=t^{n} f(x, y, z),$ 则称 $f(x, y, z)$ 为
${n}$ 次齐次函数。设 $f(x, y, z)$ 可微。

证明: $f(x, y, z)$ 为 ${n}$ 次齐次函数的充要条件是
$x \frac{\partial f}{\partial x}+y \frac{\partial f}{\partial y}+z \frac{\partial f}{\partial z}=n f(x, y, z)$。

\begin{proof}
	~
\begin{enumerate}[(1)]
	\item 必要性
	对任意固定参数 $X, Y, Z$,
	设 $x=X t,\ y=Y t,\ z=Z t$,
\begin{equation}
	f(x, y, z)=f(X t, Y t, Z t)=t^{n} f(X, Y, Z),
\end{equation}
\begin{equation}
	\frac{\dif f}{\dif t}=\frac{\partial f}{\partial x} X+\frac{\partial f}{\partial y} Y+\frac{\partial f}{\partial z} Z=n t^{n-1} f(X, Y, Z),
\end{equation}
\begin{equation}
	\frac{\partial f}{\partial x} X t+\frac{\partial f}{\partial y} Y t+\frac{\partial f}{\partial z} Z t=n t^{n} f(X, Y, Z),
\end{equation}
即
\begin{equation}
	x \frac{\partial f}{\partial x}+y \frac{\partial f}{\partial y}+z \frac{\partial f}{\partial z}=n f(x, y, z).
\end{equation} 


\item 充分性
对任意固定 $x,y,z$

$\text { 设 } X=x t,\ Y=y t,\ Z=z t$,
\begin{align}
f(X, Y, Z)=& f(t x, t y, t z), \\
\frac{\dif f(X, Y, Z)}{\dif t} &=\frac{\partial f}{\partial X} x+\frac{\partial f}{\partial Y} y+\frac{\partial f}{\partial Z} z, \\
&=\frac{\partial f}{\partial X} \frac{X}{t}+\frac{\partial f}{\partial Y} \frac{Y}{t}+\frac{\partial f}{\partial Z} \frac{Z}{t}, \\
&=\frac{1}{t} n f(X, Y, Z). \\
\therefore f(X, Y, Z) &=C t^{n}.
\end{align}
令 $t=1$ 可得 $C=f(x, y, z)$,
即 $f(x t, y t, z t)=t^{n} f(x, y, z)$.
\end{enumerate}

\end{proof}




\subsection{3}

设 $f(x, y)$ 在区域 ${D}$ 上满足 $f_{x}(x, y) \equiv 0$。

问: $f(x, y)$ 在 ${D}$ 上能否表示为 $\varphi(y)$。

不能。可举反例: 
\begin{equation}
	f(x, y)=
	\begin{cases}
		\operatorname{sgn}(x) y^{2}\qquad&(x \neq 0, y>0)\\
		0\qquad&(y \leq 0)
	\end{cases}
\end{equation}

若对于区域内任意 $y=y_{0},\ x$ 的定义域是连续的(凸区域即满足此条件), 则可由
$f(x, y)=f\left(x_{0}, y\right)+f_{x}\left(x_{0}+\theta \Delta x, y\right) \Delta x=f\left(x_{0}, y\right)$ 得 $f(x, y)$ 在 ${D}$ 上能表示为 $\varphi(y)$。
