\subsection{作业}

16.1.2 (1) (2) (3), 16.1.3, 16.4.1 (1) (2) (3), 16.4.2 (1) (2), 16.1.7, 16.2.4, 16.2.15 (1) (2) (3), 16.3.2 (1) (3) (5) (7) (9)


\subsection{1}

设 $f_{x}(x, y)$ 在 $\left(x_{0}, y_{0}\right)$ 存在, $f_{y}(x, y)$ 在 $\left(x_{0}, y_{0}\right)$ 连续。

求证: $f(x, y)$ 在 $\left(x_{0}, y_{0}\right)$ 可微。



\subsection{2}

若函数 $f(x, y, z)$ 对任意正实数 ${t}$ 满足关系 $f(t x, t y, t z)=t^{n} f(x, y, z),$ 则称 $f(x, y, z)$ 为
${n}$ 次齐次函数。设 $f(x, y, z)$ 可微。

证明: $f(x, y, z)$ 为 ${n}$ 次齐次函数的充要条件是
$x \frac{\partial f}{\partial x}+y \frac{\partial f}{\partial y}+z \frac{\partial f}{\partial z}=n f(x, y, z)$。

\subsection{3}

设 $f(x, y)$ 在区域 ${D}$ 上满足 $f_{x}(x, y) \equiv 0$。

问: $f(x, y)$ 在 ${D}$ 上能否表示为 $\varphi(y)$。

