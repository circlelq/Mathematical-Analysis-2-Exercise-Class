
\subsection{1}

设可微函数 $u=f(x, y)$ 满足方程 $x \frac{\partial f}{\partial x}+y \frac{\partial f}{\partial y}=0.$

证明: $f(x, y)$ 在极坐标系里除原点的全空间只是 $\theta$ 的函数.

\begin{proof}
令 $x=r \cos \theta,\ y=r \sin \theta$,
\begin{equation}
	\frac{\partial f}{\partial r}=\frac{\partial f}{\partial x} \cos \theta+\frac{\partial f}{\partial y} \sin \theta=\frac{1}{r}\left(\frac{\partial f}{\partial x} r \cos \theta+\frac{\partial f}{\partial y} r \sin \theta\right)=\frac{1}{r}\left(\frac{\partial f}{\partial x} x+\frac{\partial f}{\partial y} y\right)=0.
\end{equation}

$\therefore f(x, y)$ 是 $\theta$ 的函数.

若题中 $u=f(x, y) \in C^{1}(D) ,\  D$ 为含原点的凸区域,则 $f(x, y)$ 在 $D$ 上为一常数。
由 $u=f(x, y) \in C^{1}(D)$ 可得 ${r}=0$ 时, $\frac{\partial f}{\partial r}=0$.

由有限增量定理
\begin{equation}
	f(x, y)=f(r \cos \theta, r \sin \theta)=g(r, \theta)=g_{0}+g_{r}(\lambda r, \theta) r=g_{0}=f(0,0),
\end{equation}
其中 $g(r, \theta)=f(r \cos \theta, r \sin \theta),\ \frac{\partial g}{\partial r}=\frac{\partial f}{\partial r}=0,\  {r}=0$ 时, $g(r, \theta)=g_{0}=f(0,0)$.

\end{proof}



\subsection{2}

设二元函数 $F(x, y)=f(x) g(y),$ 在极坐标系可表示为 $F(x, y)=S(r),$ 求 $F(x, y)$ .


{\sf\bfseries{解}}: 令 $x=r \cos \theta, y=r \sin \theta$

$\because F(x, y)=S(r)$,

$\therefore $
\begin{equation}
	\frac{\partial F}{\partial \theta}=-\frac{\partial F}{\partial x} r \sin \theta+\frac{\partial F}{\partial y} r \cos \theta=-y \frac{\partial F}{\partial x}+x \frac{\partial F}{\partial y}=0.
\end{equation}

即 $y f^{\prime}(x) g(y)=x f(x) g^{\prime}(y)$
\begin{equation}
	\frac{f^{\prime}(x)}{x f(x)}=\frac{g^{\prime}(y)}{y g(y)}=C,	
\end{equation}
可得 $f(x)=C_{1} \me^{\frac{C}{2} x^{2}},\ g(y)=C_{2} \me^{\frac{C}{2} y^{2}}$,\ $F(x, y)=f(x) g(y)=C_{3} \me^{C_{4}\left(x^{2}+y^{2}\right)}$.




\subsection{3}

函数 ${u}$ 满足 $u u_{x y}=u_{x} u_{y}.$


求证: $u(x, y)=f(x) g(y)$.
\begin{proof}
由已知
\begin{align}
\frac{\partial u_{x}}{u_{x} \partial y}&=\frac{\partial u}{u \partial y},\\
\frac{\partial \ln u_{x}}{\partial y}&=\frac{\partial \ln u}{\partial y}, \\
\ln u_{x}&=\ln u+c(x), \\
u_{x}&=u C(x),\\
\frac{\partial \ln u}{\partial x}&=C(x),\\
\ln u&=F(x)+G(y),\\
 u(x, y)&=f(x) g(y).
\end{align}
\end{proof}


