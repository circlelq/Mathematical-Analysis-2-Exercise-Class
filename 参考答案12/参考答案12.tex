\documentclass[12pt]{article}
\usepackage{natbib}
\usepackage{url}
\usepackage{amsmath}
\usepackage{graphicx}
\usepackage{esint} % \oiint
\graphicspath{{images/}}
\usepackage{parskip}
\usepackage{fancyhdr}
\usepackage{commath}%定义d
\usepackage[UTF8]{ctex}
\usepackage{geometry}
\usepackage{titlesec}
\usepackage{caption}
\usepackage{paralist}
\usepackage{multirow}
\usepackage{booktabs} % To thicken table lines
\usepackage{diagbox}
\usepackage{bm}
\usepackage{authblk}
\usepackage{indentfirst}
\usepackage{float}
\usepackage{amsthm}
\usepackage{fontspec}
\usepackage{color}
\usepackage[perpage]{footmisc}%脚注每页清零
%\usepackage{txfonts} %设置字体为times new roman
\usepackage{lettrine}
\usepackage{nameref}
%\usepackage[nottoc]{tocbibind}
\usepackage{amssymb}%font
\usepackage{lipsum}%make test words
\usepackage{picinpar}%words around the picture
\usepackage[all]{xy}%draw arrow
\usepackage{asymptote}%draw picture
\usepackage{fontawesome}
\usepackage{titletoc}
\usepackage{fourier-orns}
\usepackage{nameref}
% \geometry{bottom=4cm}
\renewcommand{\proofname}{\indent \sf \bfseries{证明}}

\newcommand{\sgn}{\mathop{\mathrm{sgn}}}
\ctexset{today=old}
\renewcommand\contentsname{Contents}
\geometry{bottom=3cm,a4paper,left=2.5cm,right=2.5cm,top=3cm}

\pagestyle{fancy}
\fancyhf{}
\cfoot{\thepage}
%\lhead{\bfseries\rightmark}

%定义常数i、e、积分符号d
\newcommand\mi{\mathrm{i}}
\newcommand\me{\mathrm{e}}

%\setmainfont{Times New Roman}

%\ctexset{today=small}%日期类型设置

% ======================================
% = Color de la Universidad de Sevilla =
% ======================================
\usepackage{tikz}
\definecolor{PKUred}{RGB}{126,24,28}

%超链接设置
\usepackage[breaklinks,colorlinks,linkcolor=PKUred,citecolor=PKUred,urlcolor=black,pagebackref,bookmarksnumbered]{hyperref}
\usepackage{cleveref}

\newcommand{\hsp}{\hspace{20pt}}

\titleformat{\section}{\LARGE\sffamily\bfseries}{\hspace{-15pt}\textcolor{PKUred}{\vrule width 4pt height 20pt}\hsp\arabic{section}}{10em}{}

\titleformat{\subsection}{\Large\bfseries}{}{-7pt}{}

\titlecontents{section}[0pt]{\addvspace{1pc}%
    \sffamily}%
{\contentsmargin{0pt}%
    \bfseries\makebox[0pt][r]{\Large\thecontentslabel\enspace}%
}{}
{} [\addvspace{1pt}]


\titlecontents{subsection}
[0.2em] % ie, 1.5em (chapter) + 2.3em
{}{} {} {\titlerule*[1pc]{.}\contentspage}



\iffalse
    \renewcommand*\footnoterule{%
        \vspace*{-3pt}%
        {\color{PKUred}\hrule width 2in height 0.4pt}%
        \vspace*{2.6pt}%
    }
\fi
\renewcommand*\headrule{%
    {\color{PKUred}\hrule width \textwidth height 0pt}%
    \vspace*{2.6pt}%
}


%% Color the bullets of the itemize environment and make the symbol of the third
%% level a diamond instead of an asterisk.
%h\renewcommand*\textbullet{\dag}
\renewcommand*\labelitemi{\color{PKUred}\textbullet}
\renewcommand*\labelitemii{\color{PKUred}--}
\renewcommand*\labelitemiii{\color{PKUred}$\diamond$}
\renewcommand*\labelitemiv{\color{PKUred}\textperiodcentered}



%%% Equation and float numbering
\numberwithin{equation}{section}		% Equationnumbering: section.eq#
\numberwithin{figure}{section}			% Figurenumbering: section.fig#
\numberwithin{table}{section}				% Tablenumbering: section.tab#

\setcounter{secnumdepth}{2}
\setlength{\parindent}{2em}


%pdf文件设置
\hypersetup{
	pdfauthor={袁磊祺},
	pdftitle={参考答案12}
}

\title{
		\vspace{-1in} 	
		\usefont{OT1}{bch}{b}{n}
		\normalfont \normalsize \textsc{\LARGE Peking University}\\[1cm] % Name of your university/college \\ [25pt]
		\horrule{0.5pt} \\[0.5cm]
		\huge \bfseries{参考答案12} \\
		\horrule{2pt} \\[0.5cm]
}
\author{
		\normalfont 								\normalsize
		\href{https://github.com/circlelq/Mathematical-Analysis-2-Exercise-Class}{\faGithub}
		\quad\href{mailto:yuanlqpku@163.com}{\faEnvelope}  \quad \emph{袁磊祺}  \\	\normalsize
        \today
}
\date{}

\begin{document}


\maketitle

\subsection{21.2.1}

计算下列第一型曲线积分:

\subsubsection{(3)}

$\int_C xyz \dif s,\ C$为螺线:

$x = a\cos t,\quad y = a \sin t,\quad z = bt,\quad  0<a<b,\quad 0 \leq t \leq 2\pi$.
\begin{equation}
	\dif s=\sqrt{a^{2}+b^{2}} \dif t,
\end{equation}
\begin{equation}
	\begin{aligned}
		\int_{c} x y z \dif s & =  \int_{0}^{2 \pi} a^{2} b \sqrt{a^{2}+b^{2}}t \sin t \cos t  \dif t \\
		                      & =  a^{2} b \sqrt{a^{2}+b^{2}} \int_{0}^{2 x} t \sin t \cos t \dif t   \\
		                      & = -\frac{a^{2} b \sqrt{a^{2}+b^{2}} \pi}{2}
	\end{aligned}.
\end{equation}

\subsubsection{(4)}

$\int_C \sqrt{x^2+y^2} \dif s,\ C$为圆周$x^2+y^2 = ax$.

设
\begin{equation}
	\left\{\begin{array}{l}
		x=\frac{a}{2} \cos \theta+\frac{a}{2} \\
		y=\frac{a}{2} \sin \theta
	\end{array}\right.
\end{equation}
$\dif s=\frac{a}{2} \dif \theta$
\begin{equation}
	\begin{aligned}
		\int_{0}^{2 \pi} a \sqrt{\frac{1}{2} \cos \theta+\frac{1}{2}} \frac{a}{2} \dif \theta = & \  \frac{a^{2}}{2} \int_{0}^{2 \pi}\left|\cos \frac{\theta}{2}\right| \dif \theta \\
		=                                                                                       & \ 2 a^{2}
	\end{aligned}
\end{equation}

\subsubsection{(6)}

$\int_C xy \dif s,\ C$为球面$x^2+y^2+z^2=a^2$与平面$x+y+z=0$之交线。
\begin{equation}
	\begin{aligned}
		\int_{c} x y \dif s & =\int_{c} \frac{(x+y+z)^{2}-\left(x^{2}+y^{2}+z^{2}\right)}{6} \dif s \\
		                    & =\int_{c} \frac{0-a^{2}}{6} \dif s                                    \\
		                    & =-\frac{\pi a^{3}}{3}
	\end{aligned}
\end{equation}

\subsection{21.2.4}

\subsubsection{方法一}

换用极坐标有
\begin{equation}
	\begin{aligned}
		u(x,y) & = - \frac{\rho_0}{2} \oint\ln{\left(R^2+r^2-2Rr\cos\varphi\right)}R\dif \varphi                                                                           \\
		       & = - \rho_0R \int_0^{\pi}\ln{R^2\left[1+ \left(\frac{r^2}{R^2}\right) -2\frac{r}{R}\cos\varphi\right]}\dif \varphi                                         \\
		       & = - \rho_0R \int_0^{\pi}\ln{R^2}\dif \varphi - \rho_0R \int_0^{\pi}\ln{\left[1+ \left(\frac{r^2}{R^2}\right) -2\frac{r}{R}\cos\varphi\right]}\dif \varphi \\
		       & = - 2\pi R\rho_0\ln R - \rho_0R \int_0^{\pi}\ln{\left[1+ \left(\frac{r^2}{R^2}\right) -2\frac{r}{R}\cos\varphi\right]}\dif \varphi
	\end{aligned}
\end{equation}
设$a=\frac{r}{R}$,
\begin{equation}
	J(a) = \int^{\pi}_0 \ln{\left(a^2-2a\cos\varphi+1\right)}\dif \varphi
\end{equation}
对$J$求导
\begin{equation}
	\begin{aligned}
		J'(a) & = 2 \int^{\pi}_0 \frac{(a-\cos \varphi)\dif \varphi}{1-2a\cos \varphi + a^2} = \left.\left(\dfrac{x}{2a}+\dfrac{\left(a^2-1\right)\arctan\left(\frac{\left|a+1\right|\tan\left(\frac{x}{2}\right)}{\left|a-1\right|}\right)}{a\left|a-1\right|\left|a+1\right|}\right)\right|^{\pi}_0 \\
		      & = \begin{cases}
			\frac{\pi}{a}, & a\geq 1 \\
			0,             & a<1
		\end{cases}\end{aligned}
\end{equation}
又$J(0) = 0$,所以
\begin{equation}
	u(x, y)=\left\{\begin{array}{ll}
		2 \pi \rho_{0} R\ln \frac{1}{R},  & r=\sqrt{x^{2}+y^{2}}<R \\
		2 \pi \rho_{0} R \ln \frac{1}{r}, & r=\sqrt{x^{2}+y^{2}}>R
	\end{array}\right.
\end{equation}


\subsubsection{方法二}

此题可看作无穷长带电圆柱,可以参考电磁学教程。根据对称性和高斯定理可得电场分布是:圆柱外的电场径向向外,圆柱内无电场,圆柱电荷面密度
\begin{equation}
	\sigma_e = 2\pi \varepsilon_0 \rho_0.
\end{equation}
当$\sqrt{x^2+y^2}>R$时,等效为一无穷长直线产生的电场,线密度
\begin{equation}
	\eta_e = 2\pi R \sigma_e =4\pi^2R \varepsilon_0 \rho_0.
\end{equation}
产生的电场为
\begin{equation}
	E(x,y) = \frac{\eta_e}{2\pi \varepsilon_0 \sqrt{x^2+y^2}} = \frac{2\pi R  \rho_0}{  \sqrt{x^2+y^2}}
\end{equation}
电势
\begin{equation}
	u(x,y) = - \int E \dif \sqrt{x^2+y^2} = -2\pi R  \rho_0 \ln\left(\sqrt{x^2+y^2}\right).
\end{equation}
当$\sqrt{x^2+y^2}<R$时,由于内部无电场,则
\begin{equation}
	u(x,y) = \oint_L \rho_0 \ln \frac{1}{r} \dif s = -2\pi R  \rho_0 \ln R.
\end{equation}


\subsection{22.2.2}

\begin{align}
	\vec{r}           & =((b+a \cos \psi) \cos \varphi,(b+a \cos \psi) \sin \varphi, a \sin \psi) \\
	\vec{r}_{\psi}    & =(-a \sin \psi \cos \varphi,-a \sin \psi \sin \varphi, a \cos \psi)       \\
	\vec{r}_{\varphi} & =(-(b+a \cos \psi) \sin \varphi,(b+a \cos \psi) \cos \varphi, 0)
\end{align}
\begin{equation}
	E=\vec{r}_{\psi} \cdot \vec{r}_{\psi}=a^{2}, \quad F=\vec{r}_{\psi} \cdot \vec{r}_{\varphi}=0, \quad G=\vec{r}_{\varphi} \cdot \vec{r}_{\varphi}=(b+\cos \psi)^{2}
\end{equation}
\begin{equation}
	\begin{aligned}
		S & =\int_{\varphi_{1}}^{\varphi_{2}} \dif \varphi  \int_{\psi_{1}}^{\psi_{2}} \sqrt{E G-F^{2}} \dif \psi                                                         \\
		  & =\int_{\varphi_{1}}^{\varphi_{2}} \dif \varphi \int_{\psi_{1}}^{\psi_{2}} a(b+a \cos \psi) \dif \psi                                                          \\
		  & =a b\left(\varphi_{2}-\varphi_{1}\right)\left(\psi_{2}-\psi_{1}\right)+a^{2}\left(\varphi_{2}-\varphi_{1}\right) \int_{\psi_1}^{\psi_{2}} \cos \psi \dif \psi \\
		  & =a b\left(\varphi_{2}-\varphi_{1}\right)\left(\psi_{2}-\psi_{1}\right)+a^{2}\left(\varphi_{2}-\varphi_{1}\right)\left(\sin \psi_{2}-\sin \psi_{1}\right)
	\end{aligned}
\end{equation}
\begin{equation}
	S_{\text{all}} = 4\pi^2 ab.
\end{equation}


\subsection{22.2.5}

\begin{equation}
	\frac{\partial z}{\partial x}=\sqrt{\frac{y}{2 x}} \quad \frac{\partial z}{\partial y}=\sqrt{\frac{x}{2y}}
\end{equation}

\begin{equation}
	\begin{aligned}
		\int_{0}^{1} \dif x \int_{1- x}^{1} \sqrt{\frac{y}{2 x}+\frac{x}{2 y}+1} \dif y= & \int_{0}^{1} \dif x \int_{1- x}^{1} \frac{x+y}{\sqrt{2 x y}} \dif y                                                                     \\
		=                                                                                & \int_{0}^{1} \dif x \int_{1- x}^{1}\left(\sqrt{\frac{x}{2 y}}+\sqrt{\frac{y}{2 x}}\right) \dif y                                        \\
		=                                                                                & \left.\int_{0}^{1}\left(\sqrt{\frac{x}{2}} 2 \sqrt{y}+\sqrt{\frac{1}{2 x}} \frac{2}{3} y^{\frac{3}{2}}\right)\right|_{1- x} ^{1} \dif x \\
		=                                                                                & \frac{4\sqrt{2}}{3} - \frac{\sqrt{2}}{4}\pi
	\end{aligned}
\end{equation}


\subsection{22.2.6}

\begin{equation}
	3\left(x^{2}+y^{2}\right)=(2 a-x-y)^{2} \Rightarrow x^{2}+y^{2}-x y+2 a(x+y)=2 a^{2}
\end{equation}
\begin{align}
	\text{平面部分}\quad S_{1} & =\iint_{D} \sqrt{1+ z_x^{2}+z_{y}^{2}} \dif x \dif y=2 \iint_{D} \dif x \dif y        \\
	\text{曲面部分}\quad S_{2} & = \iint_{D} \sqrt{1+ z_x^{2}+z_{y}^{2}}\dif x \dif y=\sqrt{3} \iint_{D} \dif x \dif y
\end{align}
做变量替换
\begin{equation}
	u=\frac{x+y}{\sqrt{2}}, \quad v=\frac{x-y}{\sqrt{2}} \Rightarrow \frac{(u+2 \sqrt{2} a)^{2}}{12 a^{2}}+\frac{v^{2}}{4 a^{2}}=1
\end{equation}
$x,y$平面内的面积
\begin{equation}
	\iint_{D} \dif x \dif y=4 \sqrt{3} x a^{2}
\end{equation}
总面积
\begin{equation}
	S=4\pi a^{2}(2 \sqrt{3}+3)
\end{equation}
体积
\begin{equation}
	V=\frac{1}{3} S_{2} \cdot H=\frac{8 \sqrt{3}}{3} \pi a^{3}
\end{equation}


\subsection{22.2.7}

\begin{equation}
	\frac{x^{2}}{a^{2}}+\frac{y^{2}}{b^{2}}+\frac{z^{2}}{c^{2}}=1
\end{equation}
\begin{equation}
	\frac{\partial z}{\partial x} = -\frac{c}{a^{2}} \frac{x}{z},\quad \frac{\partial z}{\partial y}=-\frac{c^{2}}{b^{2}} \frac{y}{z}
\end{equation}
\begin{equation}
	{\sqrt{1+\left(\frac{\partial z}{\partial x}\right)^{2}+\left(\frac{\partial z}{2 y}\right)^{2}}}=\sqrt{1+\frac{c^{2}}{1-\frac{x^{2}}{a^{2}}-\frac{y^{2}}{b^{2}}}\left(\frac{x^{2}}{a^{4}}+\frac{y^{2}}{b^{4}}\right)} = A
\end{equation}
\begin{equation}
	S_{1}=\iint_{D_1} A \dif x \dif y \quad S_{2}= \iint_{D_{2}} A \dif x \dif y
\end{equation}
其中
\begin{equation}
	\begin{array}{l}
		D_{1}=\left\{(x, y)\left| \frac{x^{2}}{a^{2}}+\frac{y^{2}}{b^{2}}+\frac{z^{2}}{c^{2}}=1,\quad  lx+m y+n z \geqslant p\right\} \right. \\
		D_{2}=\left\{(x, y)\left|\frac{x^{2}}{a^{2}}+\frac{y^{2}}{b^{2}}+\frac{z^{2}}{c^{2}}=1,\quad  lx+m y+n z \leqslant p\right\}\right.
	\end{array}
\end{equation}


\subsection{22.3.1}

求下列积分

\subsubsection{(1)}

\begin{equation}
	z =\sqrt{x^{2}+y^{2}}
	\Rightarrow z_x=\frac{x}{\sqrt{x^{2}+y^{2}}}, \quad z_{y}=\frac{y}{\sqrt{x^{2}-2 y^{2}}}
\end{equation}
\begin{equation}
	\sqrt{z_x^{2}+z_{y}^{2}+1}=\sqrt{2}.
\end{equation}
\begin{equation}
	\begin{aligned}
		\iint_{S}\left(x^{2}+y^{2}\right) \dif s = & \iint_{x^{2}+y^{2} \leq 1} \sqrt{2}\left(x^{2}+y^{2}\right) \dif x \dif y+\iint_{x^{2}+y^{2} \leq 1}\left(x^{2}+y^{2}\right) \dif x \dif y \\
		=                                          & \int_{0}^{2 z} \dif \theta \int_{0}^{1}(\sqrt{2}+1) r^{3} \dif r                                                                           \\
		=                                          & \frac{\pi}{2}(\sqrt{2}+1)
	\end{aligned}
\end{equation}


\subsubsection{(2)}

\begin{equation}
	\frac{\partial z}{\partial x}=2 x \quad \frac{\dif z}{\partial y}=2 y
\end{equation}
\begin{equation}
	\begin{aligned}
		\iint_{s}\left|z x^{3} y^{2}\right| \dif s= & \iint_{x^{2}+y^{2} \leq 1}\left|x^{3}-y^{2}\right|\left(x^{2}+y^{2}\right) \sqrt{4 x^{2}+4 y^{2}+1} \dif x \dif y     \\
		=                                           & \int_{0}^{2 \pi} \dif \theta \int_{0}^{1} r^{8} \sqrt{4 r^{2}+1}\left|\cos ^{3} \theta \sin ^{2} \theta\right| \dif r \\
		=                                           & \frac{8}{15}\times \frac{40262 \sqrt{5}+21\,\mathrm{arcsinh}(2)}{393216}                                              \\
		=                                           & \frac{40262 \sqrt{5}+21\,\mathrm{arcsinh}(2)}{737280}
	\end{aligned}
\end{equation}



\subsubsection{(3)}

\begin{equation}
	x_{u}=\cos v, \quad y_{u}=\sin v \quad z_{u}=0
\end{equation}
\begin{equation}
	x_{v}=-u \sin v,\quad y_{v}=u\cos v, \quad z_v=1
\end{equation}
\begin{equation}
	E=\vec{r}_{u} \cdot \vec{r}_{u}=1 \quad F=\vec{r}_{u} \cdot \vec{r}_{v}=0 \quad G =\vec{r}_{v} \cdot \vec{r}_{v}=u^{2}+1
\end{equation}
\begin{equation}
	\begin{aligned}
		\iint_{S} z^{2} \dif s = & \int_{0}^{a} \dif u \int_{0}^{2 \pi} \sqrt{u^{2}+1} v^{2} \dif v                   \\
		=                        & \frac{8 \pi^{3}}{3} \int_{0}^{a} \sqrt{u^{2}+1} \dif u                             \\
		=                        & \frac{4 \pi^{3}}{3}\left[a \sqrt{1+a^{2}}+\ln \left(a+\sqrt{1+a^{2}}\right)\right] \\
	\end{aligned}
\end{equation}


\subsubsection{(4)}

\begin{equation}
	\begin{aligned}
		\iint_{S}\left(x^{2}+y^{2}\right) \dif s= & \frac{2}{3} \iint_{S} x^{2}+y^{2}+z^{2} \dif s \\
		=                                         & \frac{8}{3} \pi R^{4}
	\end{aligned}
\end{equation}


\end{document}
