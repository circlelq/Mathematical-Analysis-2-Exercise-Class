\documentclass[11pt,UTF8]{article}
\usepackage{natbib}
\usepackage{url}
\usepackage{amsmath}
\usepackage{graphicx}
\usepackage{esint} % \oiint
\graphicspath{{images/}}
\usepackage{parskip}
\usepackage{fancyhdr}
\usepackage{commath}%定义d
\usepackage[UTF8]{ctex}
\usepackage{geometry}
\usepackage{titlesec}
\usepackage{caption}
\usepackage{paralist}
\usepackage{multirow}
\usepackage{booktabs} % To thicken table lines
\usepackage{diagbox}
\usepackage{bm}
\usepackage{authblk}
\usepackage{indentfirst}
\usepackage{float}
\usepackage{amsthm}
\usepackage{fontspec}
\usepackage{color}
\usepackage[perpage]{footmisc}%脚注每页清零
%\usepackage{txfonts} %设置字体为times new roman
\usepackage{lettrine}
\usepackage{nameref}
%\usepackage[nottoc]{tocbibind}
\usepackage{amssymb}%font
\usepackage{lipsum}%make test words
\usepackage{picinpar}%words around the picture
\usepackage[all]{xy}%draw arrow
\usepackage{asymptote}%draw picture
\usepackage{fontawesome}
\usepackage{titletoc}
\usepackage{fourier-orns}
\usepackage{nameref}
% \geometry{bottom=4cm}
\renewcommand{\proofname}{\indent \sf \bfseries{证明}}

\newcommand{\sgn}{\mathop{\mathrm{sgn}}}
\ctexset{today=old}
\renewcommand\contentsname{Contents}
\geometry{bottom=3cm,a4paper,left=2.5cm,right=2.5cm,top=3cm}

\pagestyle{fancy}
\fancyhf{}
\cfoot{\thepage}
%\lhead{\bfseries\rightmark}

%定义常数i、e、积分符号d
\newcommand\mi{\mathrm{i}}
\newcommand\me{\mathrm{e}}

%\setmainfont{Times New Roman}

%\ctexset{today=small}%日期类型设置

% ======================================
% = Color de la Universidad de Sevilla =
% ======================================
\usepackage{tikz}
\definecolor{PKUred}{RGB}{126,24,28}

%超链接设置
\usepackage[breaklinks,colorlinks,linkcolor=PKUred,citecolor=PKUred,urlcolor=black,pagebackref,bookmarksnumbered]{hyperref}
\usepackage{cleveref}

\newcommand{\hsp}{\hspace{20pt}}

\titleformat{\section}{\LARGE\sffamily\bfseries}{\hspace{-15pt}\textcolor{PKUred}{\vrule width 4pt height 20pt}\hsp\arabic{section}}{10em}{}

\titleformat{\subsection}{\Large\bfseries}{}{-7pt}{}

\titlecontents{section}[0pt]{\addvspace{1pc}%
    \sffamily}%
{\contentsmargin{0pt}%
    \bfseries\makebox[0pt][r]{\Large\thecontentslabel\enspace}%
}{}
{} [\addvspace{1pt}]


\titlecontents{subsection}
[0.2em] % ie, 1.5em (chapter) + 2.3em
{}{} {} {\titlerule*[1pc]{.}\contentspage}



\iffalse
    \renewcommand*\footnoterule{%
        \vspace*{-3pt}%
        {\color{PKUred}\hrule width 2in height 0.4pt}%
        \vspace*{2.6pt}%
    }
\fi
\renewcommand*\headrule{%
    {\color{PKUred}\hrule width \textwidth height 0pt}%
    \vspace*{2.6pt}%
}


%% Color the bullets of the itemize environment and make the symbol of the third
%% level a diamond instead of an asterisk.
%h\renewcommand*\textbullet{\dag}
\renewcommand*\labelitemi{\color{PKUred}\textbullet}
\renewcommand*\labelitemii{\color{PKUred}--}
\renewcommand*\labelitemiii{\color{PKUred}$\diamond$}
\renewcommand*\labelitemiv{\color{PKUred}\textperiodcentered}



%%% Equation and float numbering
\numberwithin{equation}{section}		% Equationnumbering: section.eq#
\numberwithin{figure}{section}			% Figurenumbering: section.fig#
\numberwithin{table}{section}				% Tablenumbering: section.tab#

\setcounter{secnumdepth}{2}
\setcounter{tocdepth}{2}
\title{思考题}								% Title
\author{\emph{袁磊祺}}								% Author
\date{\ \today}											% Date

%pdf文件设置
\hypersetup{
	pdfauthor={袁磊祺},
	pdftitle={思考题}
}


\makeatletter
\let\athetitle\@title
\let\thetitle\@title
\let\theauthor\@author
\let\thedate\@date
\makeatother



\begin{document}

% \input{setc.tex}

\begin{titlepage}
	\centering
	\vspace*{-2 cm}
	\includegraphics[scale = 0.5]{fig/pkured.pdf}\\[1.0 cm]	% University Logo
	\textsc{\LARGE Peking University}\\[4.0 cm]	% University Name
	\textsc{\Large College of Engineering}\\[0.7 cm]				% Course Code

	\rule{\linewidth}{0.5 mm} \\[0.8 cm]
	{ \huge \bfseries \athetitle}\\[0.4 cm]
	\rule{\linewidth}{0.5 mm} \\[1 cm]
	\textsc{\large Mathematical Analysis (2)}\\[6 cm]				% Course Name
	\centering
	\large
	\href{https://github.com/circlelq/Mathematical-Analysis-2-Exercise-Class}{\faGithub}
	\quad \href{mailto:yuanlqpku@163.com}{\faEnvelope}\quad \theauthor\\[0.5 cm]
	\thedate\\[2 cm]

\end{titlepage}

%%%%%%%%%%%%%%%%%%%%%%%%%%%%%%%%%%%%%%%%%%%%%%%%%%%%%%%%%%%%%%%%%%%%%%%%%%%%%%%%%%%%%%%%%

{
\hypersetup{linkcolor=black}
\tableofcontents
}
\pagebreak

%%%%%%%%%%%%%%%%%%%%%%%%%%%%%%%%%%%%%%%%%%%%%%%%%%%%%%%%%%%%%%%%%%%%%%%%%%%%%%%%%%%%%%%%%
\renewcommand{\subsectionmark}[1]{\markright{#1}}


\section{}
\input{"../思考题1/答案1.tex"}

\section{}
\subsection{1}
设 $f(x)$ 在 $[0,1]$ 上严格单调下降,求证:
\begin{enumerate}[(1)]
	\item $\exists \theta \in(0,1)$ 使得 $\int_{0}^{1} f(x) \dif x=\theta f(0)+(1-\theta) f(1)$;
	\item $\forall c>f(0),\ \exists \theta \in(0,1)$ 使得 $\int_{0}^{1} f(x) \dif x=\theta c+(1-\theta) f(1)$.
\end{enumerate}

\begin{proof}
	~
	\begin{enumerate}[(1)]
		\item  $\int_{0}^{1} f(x) \dif x=f(0) \int_{0}^{\theta} \dif x+f(1) \int_{\theta}^{1} \dif x=\theta f(0)+(1-\theta) f(1)$.
		\item 由(1)可知 $\exists \theta_{1} \in(0,1) \int_{0}^{1} f(x) \dif x=\theta_{1} f(0)+\left(1-\theta_{1}\right) f(1)$.

		要证 $\exists \theta_{2} \in(0,1)$ 使得 $\int_{0}^{1} f(x) \dif x=\theta_{2} c+\left(1-\theta_{2}\right) f(1)$,
		
		即证$\exists \theta_{2} \in(0,1)$ 使得 $\theta_{1} f(0)+\left(1-\theta_{1}\right) f(1)=\theta_{2} c+\left(1-\theta_{2}\right) f(1)$
		取 $\theta_{2}=\frac{f(0)-f(1)}{c-f(1)} \theta_{1}$ 即可, $\theta_{2} \in(0,1)$.
	\end{enumerate}

\end{proof}

\subsection{2}

设 $F(x)$ 是 $[0,+\infty)$ 上单调增加的正值函数, $y$ 是微分方程 $y^{\prime \prime}+F(x) y=0$ 的解.

求证:
$y$ 在 $[0,+\infty)$ 上有界.

\begin{proof}
	由 $y^{\prime \prime}+F(x) y=0$ 可得

\begin{equation}
	\frac{1}{F(x)} y^{\prime} y^{\prime \prime}+y y^{\prime}=0,
\end{equation}
即
\begin{equation}
	\frac{1}{F(x)} \frac{\dif\left(y^{\prime}\right)^{2}}{\dif x}+\frac{\dif y^{2}}{\dif x}=0.
\end{equation}
\begin{align}
	y^{2}(x)&=y^{2}(0)-\int_{0}^{x} \frac{1}{F(t)} \frac{\dif \left(y^{\prime}\right)^{2}}{\dif t} \dif t \\
	&=y^{2}(0)-\frac{1}{F(0)} \int_{0}^{\xi} \frac{\dif \left(y^{\prime}\right)^{2}}{\dif t} \dif t \\
	&=y^{2}(0)-\frac{1}{F(0)}\left[\left(y^{\prime}(\xi)^{2}-y^{\prime}(0)^{2}\right]\right. \\
	&\leqslant y^{2}(0)+\frac{1}{F(0)} y^{\prime}(0)^{2}.
\end{align}
其中$\xi \in [0,x]$.


\end{proof}


\subsection{3}


求 $F(x)=\int_{0}^{x} \cos \frac{1}{t} \dif t$ 在原点的导数 $F^{\prime}(0)$.
\begin{equation}
	F^{\prime}(0)=\lim _{x \rightarrow 0} \frac{F(x)}{x}=\lim _{x \rightarrow 0} \frac{F(2 x)}{2 x}=\lim _{x \rightarrow 0} \frac{F(2 x)-F(x)}{x},
\end{equation}
\begin{align}
	F(2 x)-F(x)&=\int_{x}^{2 x} \cos \frac{1}{t} \dif t \\
	&=\int_{\frac{1}{2 x}}^{\frac{1}{x}} \frac{1}{t^{2}} \cos t \dif t \\
	&=4 x^{2} \int_{\frac{1}{2 x}}^{c} \cos t \dif t \\
	&=4 x^{2}\left(\sin c-\sin \frac{1}{2 x}\right).
\end{align}

\begin{equation}
F^{\prime}(0)=\lim _{x \rightarrow 0} \frac{F(2 x)-F(x)}{x}=\lim _{x \rightarrow 0} 4 x\left(\sin c-\sin \frac{1}{2 x}\right)=0.
\end{equation}

另:
\begin{align}
	& \int_{0}^{x} \cos \frac{1}{t} \dif t \\
	= & \int^\infty_{\frac{1}{x}} \frac{\cos u}{u^2} \dif u \\
	= & \lim_{M \to +\infty} \int^M_\frac{1}{x} \frac{\cos u}{u^2}\dif u\\
	= & x^2 \lim_{M \to +\infty} \int^\xi_\frac{1}{x} {\cos u}\dif u\quad \left(\xi \in \left[\frac{1}{x},M \right] \right)\\
	= & \mathcal{O}(x^2).
\end{align}
\begin{equation}
	F^{\prime}(0)=\lim _{x \rightarrow 0} \frac{F(x)}{x}=\lim _{x \rightarrow 0}  \mathcal{O}(x)=0.
\end{equation}


\subsection{4}

若 $f(x)$ 在 $[a,+\infty)$ 上单调下降,且积分 $\int_{a}^{+\infty} f(x) \dif x$ 收敛.

求证: $\lim \limits_{x \rightarrow \infty} x f(x)=0$.

\begin{proof}
	由收敛原理,

	当$\eta \rightarrow+\infty$ 有 
	\begin{equation}
		\int_{a}^{2 \eta} f(x) \dif x-\int_{a}^{\eta} f(x) \dif x=\int_{\eta}^{2 \eta} f(x) \dif x \rightarrow 0.
	\end{equation}
	$f(x)$ 在 $[a,+\infty)$ 上单调下降,
\begin{equation}
	\therefore \int_{\eta}^{2 \eta} f(x) \dif x \leqslant \int_{\eta}^{2 \eta} f(\eta) \dif x=f(\eta) \eta=2 \int_{\eta / 2}^{\eta} f(\eta) \dif x \leqslant 2 \int_{\eta / 2}^{\eta} f(x) \dif x.
\end{equation}
	由夹挤原理即有
	\begin{equation}
		\therefore \lim _{x \rightarrow \infty} x f(x)=0.
	\end{equation}
\end{proof}


\section{}
\subsection{1}
设 $\int_{a}^{+\infty} f(x) \dif x$ 收敛, $x f(x)$ 在 $[a,+\infty)$ 上单调下降.

求证:
\begin{enumerate}[(1)]
	\item $x f(x) \geqslant 0,\ (x \geqslant a) ;$
	\item $\lim \limits_{x \rightarrow+\infty} x f(x) \ln x=0$.
\end{enumerate}
\begin{proof}
	~
	\begin{enumerate}[(1)]
		\item $\because \int_{a}^{+\infty} f(x) \dif x$ 收敛,
		
		$\therefore H \to +\infty,\ \int_{H}^{2 H} f(x) \dif x \rightarrow 0$.
		\begin{align}
		\int_{H}^{2 H} f(x) \dif x &=\int_{H}^{2 H} x f(x) \cdot \frac{1}{x} \dif x \\
		&=H f(H) \int_{H}^{c} \frac{1}{x} \dif x+2 H f(2 H) \int_{c}^{2 H} \frac{1}{x} \dif x \\
		&=H f(H)(\ln c-\ln H)+2 H f(2 H)(\ln 2 H-\ln c) \\
		&=(H f(H)-2 H f(2 H)) \ln c-H f(H) \ln H+2 H f(2 H) \ln 2 H \\
		&=(H f(H)-2 H f(2 H))(\ln c-\ln H)+2 H f(2 H) \ln 2.
		\end{align}
		若 $\exists x^{*}$ 使得 $x^{*} f\left(x^{*}\right)=\zeta<0$, 由单调性 $\forall x>x^{*},$ 有 $x f(x)<0$. 当$H=x^*$时,
		\begin{align}
		\int_{H}^{2 H} f(x) \dif x &=(H f(H)-2 H f(2 H))(\ln c-\ln H)+2 H f(2 H) \ln 2 \\
		& \leqslant (H f(H)-2 H f(2 H)) \ln 2+2 H f(2 H) \ln 2 \\
		&=H f(H) \ln 2 \\
		&=\zeta \ln 2.
		\end{align}
		与积分收敛矛盾,
		因此 $x f(x) \geqslant 0,(x \geqslant a)$ .

		另: 也可直接得
		\begin{align}
		\int_{H}^{2 H} f(x) \dif x &=\int_{H}^{2 H} x f(x) \cdot \frac{1}{x} \dif x \\
		& \leqslant  \int_{H}^{2 H} H f(H) \cdot \frac{1}{x} \dif x \\
		&=H f(H) \ln 2 \\
		& \leqslant  \zeta \ln 2.
		\end{align}
		\item 由收敛有 $H \rightarrow+\infty,\ \int_{H}^{H^{2}} f(x) \dif x \rightarrow 0$,\
		$x f(x) \geqslant 0(x \geqslant a)$,
		\begin{equation}
			\int_{H}^{H^{2}} f(x) \dif x=\int_{H}^{H^{2}} x f(x) \cdot \frac{1}{x} \dif x \geqslant \frac{1}{2} H^{2} f\left(H^{2}\right) \ln H^{2} \geqslant 0.
		\end{equation}
		$\therefore x f(x) \ln x \rightarrow 0$.
	\end{enumerate}

\end{proof}

\subsection{2}

设 $f(x)$ 在 $[0,+\infty)$ 上连续,且 $\lim \limits_{x \rightarrow+\infty} f(x)=f(+\infty)$ 存在.

求证:
\begin{equation}
	\int_{0}^{+\infty} \frac{f(a x)-f(b x)}{x} \dif x=[f(0)-f(+\infty)] \ln \frac{b}{a},\ (b>a>0).
\end{equation}

\begin{proof}
	
	考虑 $\int_{c_{1}}^{c_{2}} \frac{f(a x)-f(b x)}{x} \dif x,\ c_{1} \rightarrow 0,\ c_{2} \rightarrow+\infty,$

其中
\begin{equation}
	\int_{c_{1}}^{c_{2}} \frac{f(a x)}{x} \dif x=\int_{c_{1}}^{c_{2}} \frac{f(a x)}{a x} \dif \ (a x)=\int_{a c_{1}}^{a c_{2}} \frac{f(x)}{x} \dif x,
\end{equation}
同理
\begin{equation}
	\int_{c_{1}}^{c_{2}} \frac{f(b x)}{x} \dif x=\int_{b c_{1}}^{b c_{2}} \frac{f(x)}{x} \dif x,
\end{equation}
则
\begin{align}
\int_{c_{1}}^{c_{2}} \frac{f(a x)-f(b x)}{x} \dif x &=\int_{a c_{1}}^{a c_{2}} \frac{f(x)}{x} \dif x-\int_{b c_{1}}^{b c_{2}} \frac{f(x)}{x} \dif x \\
&=\int_{a c_{1}}^{b c_{1}} \frac{f(x)}{x} \dif x-\int_{a c_{2}}^{b c_{2}} \frac{f(x)}{x} \dif x \\
&=f(\xi) \int_{a c_{1}}^{b c_{1}} \frac{1}{x} \dif x-f(\eta) \int_{a c_{2}}^{b c_{2}} \frac{1}{x} \dif x \\
&=(f(\xi)-f(\eta)) \ln \frac{b}{a}.
\end{align}
其中$\xi \in\left(a c_{1}, b c_{1}\right), \eta  \in\left(a c_{2}, b c_{2}\right).$

$c_{1} \rightarrow 0^+, c_{2} \rightarrow+\infty,  \text { 即有 } $
\begin{equation}
	\int_{0}^{+\infty} \frac{f(a x)-f(b x)}{x} \dif x=[f(0)-f(+\infty)] \ln \frac{b}{a}.
\end{equation}

\end{proof}

\subsection{3}

设 $f(x)$ 在 $[a,+\infty)$ 上一致连续,且 $\int_{a}^{+\infty} f(x) \dif x$ 收敛.

求证: $\lim \limits_{x \rightarrow+\infty} f(x)=0$.

\begin{proof}
设 $f(x)$ 不收敛于 $0,$ 则
$$
\exists \varepsilon>0,\ \forall N,\ \exists x_{0}>N,\ \mathrm{s.t.} \left|f\left(x_{0}\right)\right|>\varepsilon.
$$

由一致连续性, $\exists \delta >0,$ 当$\abs{x-x_0}<\delta$时, 有$\abs{f(x)-f(x_0)}<\frac{\varepsilon}{2}$, 即 $-\frac{\varepsilon}{2}<{f(x)-f(x_0)}<\frac{\varepsilon}{2}$.

$\text { 若 } f\left(x_{0}\right)>\varepsilon, \text { 则 } f(x)>f\left(x_{0}\right)-\frac{\varepsilon}{2}>\frac{\varepsilon}{2} ;$

$\text { 若 } f\left(x_{0}\right)<-\varepsilon, \text { 则 } f(x)<f\left(x_{0}\right)+\frac{\varepsilon}{2}<-\frac{\varepsilon}{2}$.

$\text { 则有 } \int_{x_{0}}^{x_{0}+\delta} f(x) \dif x>\frac{\varepsilon \delta}{2} \text { 或 } \int_{x_{0}}^{x_{0}+\delta} f(x) \dif x<-\frac{\varepsilon \delta}{2}, \text { 与 } \int_{a}^{+\infty} f(x) \dif x \text { 收敛矛盾. }$

$\therefore \lim \limits_{x \rightarrow+\infty} f(x)=0$.

\end{proof}


\section{}

\subsection{1}

求证:

$\lim \limits_{x \rightarrow+\infty,\ y \rightarrow-\infty}[f(x)+g(y)] \text { 存在的充要条件是 } \lim \limits_{x \rightarrow+\infty} f(x) \text { 和 } \lim \limits_{y \rightarrow-\infty} g(y) \text { 同时存在. }$

\begin{proof}
	~
	\begin{enumerate}[(1)]
		\item 充分条件
		
		$\text { 设 } \lim \limits_{x \rightarrow+\infty,\ y \rightarrow-\infty}[f(x)+g(y)]=A$,

		$\forall \varepsilon>0,\ \exists N_{1}, N_{2}>0,$ 当 $x_{1}>N_{1},\ y_{i}<-N_{2},(i=1,2)$ 时有

		\begin{equation}
			\left|f\left(x_{1}\right)+g\left(y_{i}\right)-A\right|<\frac{\varepsilon}{2},
		\end{equation}
	$\therefore \forall \varepsilon>0,\ \exists N_{2}>0,\ \forall y_{1}, y_{2}<-N_{2} \text { 有 }$
	\begin{align}
		\abs{g\left(y_{1}\right)-g\left(y_{2}\right)}  &=\left|\left(f\left(x_{1}\right)+g\left(y_{1}\right)-A\right)-\left(f\left(x_{1}\right)+g\left(y_{2}\right)-A\right)\right| \\
		&<\left|f\left(x_{1}\right)+g\left(y_{1}\right)-A\right|+\left|f\left(x_{1}\right)+g\left(y_{2}\right)-A\right|<\varepsilon 
	\end{align}
	$\therefore \lim \limits_{y \rightarrow-\infty} g(y) \text { 存在 }$
	
	同理可证 $\lim \limits_{x \rightarrow+\infty} f(x)$ 存在.

	\item 必要条件
	
	若 $\lim \limits_{x \rightarrow+\infty} f(x)$ 和 $\lim \limits_{y \rightarrow-\infty} g(y)$ 同时存在,设 $\lim \limits_{x \rightarrow+\infty} f(x)=a, \quad \lim \limits_{y \rightarrow-\infty} g(y)=b,$ 则

	$\forall \varepsilon>0,\ \exists N_{1}>0,$ 当 $x>N_{1}$ 时 $|f(x)-a|<\frac{\varepsilon}{2},\ \exists N_{2}>0,$ 当 $y<-N_{2}$ 时, $|g(y)-b|<\frac{\varepsilon}{2}$

	则 $|f(x)+g(y)-(a+b)| \leqslant |f(x)-a|+|g(y)-b|<\varepsilon,$ 即有
	\begin{equation}
		\lim _{x \rightarrow+\infty,\ y \rightarrow-\infty}[f(x)+g(y)]=a+b.
	\end{equation}

	\end{enumerate}

\end{proof}

\subsection{2}


设二元函数 $f(x, y)$ 在圆周 $C:\left(x-x_{0}\right)^{2}+\left(y-y_{0}\right)^{2}=R^{2}$ 上连续.

证明: $f(x, y)$ 在 $C$
上达到上确界 ${M}$ 和下确界 ${m},$ 且取属于 $(m, M)$ 的值至少两次.

\begin{proof}
	~
	\begin{enumerate}[(1)]
		\item 在圆周 $C:\left(x-x_{0}\right)^{2}+\left(y-y_{0}\right)^{2}=R^{2}$ 上
		\begin{equation}
			\left\{\begin{array}{l}x=x(t)=x_{0}+R \cos t \\ y=y(t)=y_{0}+R \sin t\end{array},\ t \in[\theta, 2 \pi+\theta]\right.
		\end{equation}
		
		则 $f(x, y)=f(x(t), y(t))=g(t)$ 连续.

		$g(t)$ 连续性证明: $x, y$ 是 ${t}$ 的连续函数, $\therefore \forall \delta>0,\ \exists \delta_{1}, \delta_{2}>0,$ 当 $\left|t-t^{*}\right|<\delta_{1}$ 时有
		$\left|x-x^{*}\right|<\frac{\delta}{2},$ 当 $\left|t-t^{*}\right|<\delta_{2}$ 时有 $\left|y-y^{*}\right|<\frac{\delta}{2}$ .又由 $f(x, y)$ 连续性, $\forall \varepsilon>0,\ \exists \delta>0,$ 只
		要 $(x, y) \in C,\ \sqrt{\left(x-x^{*}\right)^{2}+\left(y-y^{*}\right)^{2}}<\delta,$ 就有 $\left|f(x, y)-f\left(x^{*}, y^{*}\right)\right|<\varepsilon$ .
		
		因此 $\forall \varepsilon>0,\ \exists \delta_{0}=\min \left\{\delta_{1}, \delta_{2}\right\},$ 当 $\left|t-t^{*}\right|<\delta_{0}$ 时有
		\begin{equation}
			(x, y) \in C, \sqrt{\left(x-x^{*}\right)^{2}+\left(y-y^{*}\right)^{2}}<\sqrt{\left(\frac{\delta}{2}\right)^{2}+\left(\frac{\delta}{2}\right)^{2}}<\delta,
		\end{equation}
		
		进而有 $\left|g(t)-g\left(t^{*}\right)\right|=\left|f(x, y)-f\left(x^{*}, y^{*}\right)\right|<\varepsilon, \quad \therefore g(t)$ 是 ${t}$ 的连续函数. 

		$\therefore g(t)$ 在闭区间 $[\theta, 2 \pi+\theta]$ 有界,且 $\exists t_{1},\ t_{2} \in[\theta, 2 \pi+\theta],\ \mathrm{s.t.} g\left(t_{1}\right)=m,\ g\left(t_{2}\right)=M$( 最大值与最小值定理).即 $f\left(x_{1}, y_{1}\right)=m,\  f\left(x_{2}, y_{2}\right)=M$, 其中 $x_{i}=x\left(t_{i}\right),\ y_{i}=y\left(t_{i}\right),\ (i=1,2)$.
		\item 若 $m=M$,则结论显然.
		$m \neq M$ 时,设 $g\left(\theta_{1}\right)=g\left(\theta_{1}+2 \pi\right)=f\left(x_{1}, y_{1}\right)=m,\ g\left(\theta_{2}\right)=f\left(x_{2}, y_{2}\right)=M,$ 则圆周可
		分为两段
		\begin{equation}
			\Gamma_{1}:\left\{\begin{array}{l}x=x(t)=x_{0}+R \cos t \\ y=y(t)=y_{0}+R \sin t\end{array}, t \in\left[\theta_{1}, \theta_{2}\right]\right. ; 
		\end{equation}
		\begin{equation}
			\Gamma_{2}:\left\{\begin{array}{l}x=x(t)=x_{0}+R \cos t \\ y=y(t)=y_{0}+R \sin t\end{array}, t \in\left[\theta_{2}, \theta_{1}+2 \pi\right]\right..
		\end{equation}
		
		则 $\forall \mu \in(m, M),\ \exists t_{1}^{*} \in\left(\theta_{1}, \theta_{2}\right), t_{2}^{*} \in\left(\theta_{2}, \theta_{1}+2 \pi\right),$ 满足 $g\left(t_{1}^{*}\right)=g\left(t_{2}^{*}\right)=\mu$.相应即有 $f\left(x_{1}^{*}, y_{1}^{*}\right)=f\left(x_{2}^{*}, y_{2}^{*}\right)=\mu$.
	\end{enumerate}
\end{proof}




\subsection{3}

证明: 若 $f(x, y)$ 分别对每一个变量 ${x}, {y}$ 是连续的,且对其中一个单调,则 $f(x, y)$ 是二元连续函数。

\begin{proof}
	不妨设 $f(x, y)$ 对 ${y}$ 单调。

$\forall \varepsilon>0,\ \exists \delta_{1}>0,\ \mathrm{s.t.}\  x_{0}<x<x_{0}+\delta_{1}$ 时
\begin{equation}
	\left|f\left(x, y_{0}\right)-f\left(x_{0}, y_{0}\right)\right|<\frac{\varepsilon}{4},
\end{equation}
$\forall \varepsilon>0,\ \exists \delta_{2}>0,\ \mathrm{s.t.}\ y_{0}<y<y_{0}+\delta_{2}$ 时
\begin{equation}
	\left|f\left(x_{0}, y_{0}\right)-f\left(x_{0}, y \right)\right|<\frac{\varepsilon}{4},
\end{equation}
$\forall \varepsilon>0,\ \exists \delta_{3}>0,\ \mathrm{s.t.}\ x_{0}<x<x_{0}+\delta_{3}$ 时
\begin{equation}
	\left|f\left(x, y_{0}+\delta_{2}\right)-f\left(x_{0}, y_{0}+\delta_{2}\right)\right|<\frac{\varepsilon}{4}.
\end{equation}

由以上三式可得 $\left|f\left(x, y_{0}+\delta_{2}\right)-f\left(x, y_{0}\right)\right|<\frac{3 \varepsilon}{4}$.

令 $\delta=\min \left\{\delta_{1}, \delta_{2}, \delta_{3}\right\},$ 当 $x \in\left(x_{0}, x_{0}+\delta\right), y \in\left(y_{0}, y_{0}+\delta\right)$ 时,由 ${f}$ 关于 ${y}$ 的单调性,
\begin{equation}
	\left|f(x, y)-f\left(x, y_{0}\right)\right|<\left|f\left(x, y_{0}+\delta_{2}\right)-f\left(x, y_{0}\right)\right|<\frac{3 \varepsilon}{4}.
\end{equation}
$\therefore\left|f(x, y)-f\left(x_{0}, y_{0}\right)\right|<\varepsilon$.

对于 $\left(x_{0}, y_{0}\right)$ 左方和下方的邻域内类似有同上结论。 $\therefore f(x, y)$ 二元连续。


\end{proof}


\section{}

\subsection{1}

设 $f_{x}(x, y)$ 在 $\left(x_{0}, y_{0}\right)$ 存在, $f_{y}(x, y)$ 在 $\left(x_{0}, y_{0}\right)$ 连续。

求证: $f(x, y)$ 在 $\left(x_{0}, y_{0}\right)$ 可微。

\begin{proof}
	

\begin{align}
f\left(x_{0}+\Delta x, y_{0}+\Delta y\right)-f\left(x_{0}, y_{0}\right) =&f\left(x_{0}+\Delta x, y_{0}+\Delta y\right)-f\left(x_{0}+\Delta x, y_{0}\right)\\
&+f\left(x_{0}+\Delta x, y_{0}\right)-f\left(x_{0}, y_{0}\right) \\
=&f_y\left(x_{0}+\Delta x, y_{0}+\theta \Delta y\right) \Delta y+f_x\left(x_{0}, y_{0}\right) \Delta x+o(\Delta x) \\
=&f_y\left(x_{0}, y_{0}\right) \Delta y+\alpha \Delta y+f_x\left(x_{0}, y_{0}\right) \Delta x+o(\Delta x),
\end{align}
其中$\alpha=f_y\left(x_{0}+\Delta x, y_{0}+\theta \Delta y\right)-f_y\left(x_{0}, y_{0}\right), \quad 0<\theta<1$ 

当$\Delta x, \Delta y \rightarrow 0 \text { 时, } \alpha \rightarrow 0 $,

$\therefore  \lim \limits_{\Delta x \rightarrow 0, \Delta y \rightarrow 0} \alpha \Delta y / \sqrt{\Delta x^{2}+\Delta y^{2}}=0 $.

$\therefore \alpha \Delta y=o\left(\sqrt{\Delta x^{2}+\Delta y^{2}}\right) $.

$\therefore f\left(x_{0}+\Delta x, y_{0}+\Delta y\right)-f\left(x_{0}, y_{0}\right)=f_y\left(x_{0}, y_{0}\right) \Delta y+f_x\left(x_{0}, y_{0}\right) \Delta x+o\left(\sqrt{\Delta x^{2}+\Delta y^{2}}\right)$.

\end{proof}


附: 二元函数 $f(x, y)$ 在 $\left(x_{0}, y_{0}\right)$ 连续,则在 $\left(x_{0}, y_{0}\right)$ 临域内由
\begin{align}
\left|f\left(x_{0}+\Delta x, y_{0}+\Delta y\right)-f\left(x_{0}+\Delta x, y_{0}\right)\right| \leq &\left|f\left(x_{0}+\Delta x, y_{0}+\Delta y\right)-f\left(x_{0}, y_{0}\right)\right| \\
&+\left|f\left(x_{0}+\Delta x, y_{0}\right)-f\left(x_{0}, y_{0}\right)\right|
\end{align}
可得 $f\left(x_{0}+\Delta x, y\right)$ 在 $\left(x_{0}+\Delta x, y_{0}\right)$ 关于 ${y}$ 连续.

\subsection{2}

若函数 $f(x, y, z)$ 对任意正实数 ${t}$ 满足关系 $f(t x, t y, t z)=t^{n} f(x, y, z),$ 则称 $f(x, y, z)$ 为
${n}$ 次齐次函数。设 $f(x, y, z)$ 可微。

证明: $f(x, y, z)$ 为 ${n}$ 次齐次函数的充要条件是
$x \frac{\partial f}{\partial x}+y \frac{\partial f}{\partial y}+z \frac{\partial f}{\partial z}=n f(x, y, z)$。

\begin{proof}
	~
\begin{enumerate}[(1)]
	\item 必要性
	对任意固定参数 $X, Y, Z$,
	设 $x=X t,\ y=Y t,\ z=Z t$,
\begin{equation}
	f(x, y, z)=f(X t, Y t, Z t)=t^{n} f(X, Y, Z),
\end{equation}
\begin{equation}
	\frac{\dif f}{\dif t}=\frac{\partial f}{\partial x} X+\frac{\partial f}{\partial y} Y+\frac{\partial f}{\partial z} Z=n t^{n-1} f(X, Y, Z),
\end{equation}
\begin{equation}
	\frac{\partial f}{\partial x} X t+\frac{\partial f}{\partial y} Y t+\frac{\partial f}{\partial z} Z t=n t^{n} f(X, Y, Z),
\end{equation}
即
\begin{equation}
	x \frac{\partial f}{\partial x}+y \frac{\partial f}{\partial y}+z \frac{\partial f}{\partial z}=n f(x, y, z).
\end{equation} 


\item 充分性
对任意固定 $x,y,z$

$\text { 设 } X=x t,\ Y=y t,\ Z=z t$,
\begin{align}
f(X, Y, Z)=& f(t x, t y, t z), \\
\frac{\dif f(X, Y, Z)}{\dif t} &=\frac{\partial f}{\partial X} x+\frac{\partial f}{\partial Y} y+\frac{\partial f}{\partial Z} z, \\
&=\frac{\partial f}{\partial X} \frac{X}{t}+\frac{\partial f}{\partial Y} \frac{Y}{t}+\frac{\partial f}{\partial Z} \frac{Z}{t}, \\
&=\frac{1}{t} n f(X, Y, Z). \\
\therefore f(X, Y, Z) &=C t^{n}.
\end{align}
令 $t=1$ 可得 $C=f(x, y, z)$,
即 $f(x t, y t, z t)=t^{n} f(x, y, z)$.
\end{enumerate}

\end{proof}




\subsection{3}

设 $f(x, y)$ 在区域 ${D}$ 上满足 $f_{x}(x, y) \equiv 0$。

问: $f(x, y)$ 在 ${D}$ 上能否表示为 $\varphi(y)$。

不能。可举反例: 
\begin{equation}
	f(x, y)=
	\begin{cases}
		\operatorname{sgn}(x) y^{2}\qquad&(x \neq 0, y>0)\\
		0\qquad&(y \leq 0)
	\end{cases}
\end{equation}

若对于区域内任意 $y=y_{0},\ x$ 的定义域是连续的(凸区域即满足此条件), 则可由
$f(x, y)=f\left(x_{0}, y\right)+f_{x}\left(x_{0}+\theta \Delta x, y\right) \Delta x=f\left(x_{0}, y\right)$ 得 $f(x, y)$ 在 ${D}$ 上能表示为 $\varphi(y)$。


\section{}

\subsection{1}

设可微函数 $u=f(x, y)$ 满足方程 $x \frac{\partial f}{\partial x}+y \frac{\partial f}{\partial y}=0.$

证明: $f(x, y)$ 在极坐标系里除原点的全空间只是 $\theta$ 的函数.

\begin{proof}
令 $x=r \cos \theta,\ y=r \sin \theta$,
\begin{equation}
	\frac{\partial f}{\partial r}=\frac{\partial f}{\partial x} \cos \theta+\frac{\partial f}{\partial y} \sin \theta=\frac{1}{r}\left(\frac{\partial f}{\partial x} r \cos \theta+\frac{\partial f}{\partial y} r \sin \theta\right)=\frac{1}{r}\left(\frac{\partial f}{\partial x} x+\frac{\partial f}{\partial y} y\right)=0.
\end{equation}

$\therefore f(x, y)$ 是 $\theta$ 的函数.

若题中 $u=f(x, y) \in C^{1}(D) ,\  D$ 为含原点的凸区域,则 $f(x, y)$ 在 $D$ 上为一常数。
由 $u=f(x, y) \in C^{1}(D)$ 可得 ${r}=0$ 时, $\frac{\partial f}{\partial r}=0$.

由有限增量定理
\begin{equation}
	f(x, y)=f(r \cos \theta, r \sin \theta)=g(r, \theta)=g_{0}+g_{r}(\lambda r, \theta) r=g_{0}=f(0,0),
\end{equation}
其中 $g(r, \theta)=f(r \cos \theta, r \sin \theta),\ \frac{\partial g}{\partial r}=\frac{\partial f}{\partial r}=0,\  {r}=0$ 时, $g(r, \theta)=g_{0}=f(0,0)$.

\end{proof}



\subsection{2}

设二元函数 $F(x, y)=f(x) g(y),$ 在极坐标系可表示为 $F(x, y)=S(r),$ 求 $F(x, y)$ .


{\sf\bfseries{解}}: 令 $x=r \cos \theta, y=r \sin \theta$

$\because F(x, y)=S(r)$,

$\therefore $
\begin{equation}
	\frac{\partial F}{\partial \theta}=-\frac{\partial F}{\partial x} r \sin \theta+\frac{\partial F}{\partial y} r \cos \theta=-y \frac{\partial F}{\partial x}+x \frac{\partial F}{\partial y}=0.
\end{equation}

即 $y f^{\prime}(x) g(y)=x f(x) g^{\prime}(y)$
\begin{equation}
	\frac{f^{\prime}(x)}{x f(x)}=\frac{g^{\prime}(y)}{y g(y)}=C,	
\end{equation}
可得 $f(x)=C_{1} \me^{\frac{C}{2} x^{2}},\ g(y)=C_{2} \me^{\frac{C}{2} y^{2}}$,\ $F(x, y)=f(x) g(y)=C_{3} \me^{C_{4}\left(x^{2}+y^{2}\right)}$.




\subsection{3}

函数 ${u}$ 满足 $u u_{x y}=u_{x} u_{y}.$


求证: $u(x, y)=f(x) g(y)$.
\begin{proof}
由已知
\begin{align}
\frac{\partial u_{x}}{u_{x} \partial y}&=\frac{\partial u}{u \partial y},\\
\frac{\partial \ln u_{x}}{\partial y}&=\frac{\partial \ln u}{\partial y}, \\
\ln u_{x}&=\ln u+c(x), \\
u_{x}&=u C(x),\\
\frac{\partial \ln u}{\partial x}&=C(x),\\
\ln u&=F(x)+G(y),\\
 u(x, y)&=f(x) g(y).
\end{align}
\end{proof}




\section{}
\subsection{1}

设 $\Omega \subset \mathbb{R}^{n}$ 是凸域, $f \in C^{1}\left(\Omega, \mathbb{R}^{n}\right),\  \Dif f$ 是 $\Omega$ 上的正定矩阵.

求证: $f(x)$ 是 $\Omega$ 上的一一映射.

\begin{proof}
	若 $\exists x_{1}, x_{2} \in \Omega,$ s.t. $f\left(x_{1}\right)=f\left(x_{2}\right)$,

令
\begin{equation}
	g(x)=\left(x_{2}-x_{1}\right) \cdot\left(f(x)-f\left(x_{1}\right)\right),
\end{equation}
\begin{equation}
	G(t)=g\left(x_{1}+t\left(x_{2}-x_{1}\right)\right),
\end{equation}

则 $G(0)=G(1)=0$,

$\because \Omega$ 是凸域,

$\therefore \exists \xi \in(0,1),$ s.t. $G^{\prime}(\xi)=0$,

即 $\Dif g\left(x_{1}+\xi\left(x_{2}-x_{1}\right)\right) \cdot\left(x_{2}-x_{1}\right)=0$,

$\left(x_{2}-x_{1}\right) \Dif f\left(x_{1}+\xi\left(x_{2}-x_{1}\right)\right)\left(x_{2}-x_{1}\right)^{T}=0$.

又 $\Dif f$ 是正定的,

$\therefore x_{2}-x_{1}=0$, 即 $x_{1}=x_{2}, \ f(x)$ 是 $\Omega$ 上的单射.
\end{proof}



\section{}
\subsection{1}

设 $f$ 可微,证明曲面 $f\left(\frac{z}{y}, \frac{x}{z}, \frac{y}{x}\right)=0$ 上任一点的切平面均过某一定点。

\begin{proof}
	设 $u=\frac{z}{y}, v=\frac{x}{z}, w=\frac{y}{x}$

	考虑任一点 $\left(x_{0}, y_{0}, z_{0}\right)$ 处的切平面,法向为
	\begin{equation}
		\left(f_{v}^{\prime} \frac{1}{z_{0}}-f_{w}^{\prime} \frac{y_{0}}{x_{0}^{2}}, f_{w}^{\prime} \frac{1}{x_{0}}-f_{u}^{\prime} \frac{z_{0}}{y_{0}^{2}}, f_{u}^{\prime} \frac{1}{y_{0}}-f_{v}^{\prime} \frac{x_{0}}{z_{0}^{2}}\right),
	\end{equation}
	切平面为
	\begin{equation}
		\left(f_{v}^{\prime} \frac{1}{z_{0}}-f_{w}^{\prime} \frac{y_{0}}{x_{0}^{2}}\right)\left(x-x_{0}\right)+\left(f_{w}^{\prime} \frac{1}{x_{0}}-f_{u}^{\prime} \frac{z_{0}}{y_{0}^{2}}\right)\left(y-y_{0}\right)+\left(f_{u}^{\prime} \frac{1}{y_{0}}-f_{v}^{\prime} \frac{x_{0}}{z_{0}^{2}}\right)\left(z-z_{0}\right)=0,
	\end{equation}
	即 
	\begin{equation}
		\left(f_{v}^{\prime} \frac{1}{z_{0}}-f_{w}^{\prime} \frac{y_{0}}{x_{0}^{2}}\right) x+\left(f_{w}^{\prime} \frac{1}{x_{0}}-f_{u}^{\prime} \frac{z_{0}}{y_{0}^{2}}\right) y+\left(f_{u}^{\prime} \frac{1}{y_{0}}-f_{v}^{\prime} \frac{x_{0}}{z_{0}^{2}}\right) z=0,
	\end{equation}
	必过 $(0,0,0)$ 点.
\end{proof}



\subsection{2}


求椭球面 $\frac{x^{2}}{4}+\frac{y^{2}}{6}+\frac{z^{2}}{8}=1$ 上法线与平面 $x+2 y+z=100$ 垂直的点.


考虑任一点 $\left(x_{0}, y_{0}, z_{0}\right)$ 处的法向为
\begin{equation}
	\left(\frac{x_0}{2},\frac{y_0}{3},\frac{z_0}{4}\right).
	\label{eq:21}
\end{equation}

平面 $x+2 y+z=100$ 的法向为
\begin{equation}
	(1,2,1).
	\label{eq:22}
\end{equation}

若椭球面上法线与平面垂直,则\cref{eq:21,eq:22}平行
\begin{equation}
	\frac{\frac{x_0}{2}}{1}=\frac{\frac{y_0}{3}}{2}=\frac{\frac{z_0}{4}}{1}.
	\label{eq:23}
\end{equation}
将\cref{eq:23}代入
\begin{equation}
	\frac{x^{2}}{4}+\frac{y^{2}}{6}+\frac{z^{2}}{8}=1
\end{equation}
解得
\begin{equation}
	a = \left(\frac{2}{3},2,\frac{4}{3}\right),\quad b = \left(-\frac{2}{3},-2,-\frac{4}{3}\right).
\end{equation}



\subsection{3}

求曲面 $\left\{\begin{array}{c}x+y+z=0 \\ \frac{x^{2}}{a^{2}}+\frac{y^{2}}{b^{2}}+\frac{z^{2}}{c^{2}}=1\end{array}\right.$ 交线的切线, 以及 $a, b, c$ 满足什么条件时, 交线的副法向与椭球面的法向正交?

考虑交线上任一点 $\left(x_{0}, y_{0}, z_{0}\right)$ 处的两平面的法向为
\begin{equation}
	\left(1,1,1\right),\quad \left(\frac{2x_0}{a^2},\frac{2y_0}{b^2},\frac{2z_0}{c^2}\right).
	\label{eq:31}
\end{equation}
则交线的切线为
\begin{equation}
	\left(1,1,1\right) \times \left(\frac{2x_0}{a^2},\frac{2y_0}{b^2},\frac{2z_0}{c^2}\right) = \left(\frac{2z_0}{c^2}-\frac{2y_0}{b^2},\frac{2x_0}{a^2}-\frac{2z_0}{c^2},\frac{2y_0}{b^2}-\frac{2x_0}{a^2}\right).
\end{equation}

由于交线在平面上,所以交线的副法向为
\begin{equation}
	(1,1,1),
\end{equation}
与椭球面的法向正交则
\begin{equation}
	\left(1,1,1\right) \cdot \left(\frac{2x_0}{a^2},\frac{2y_0}{b^2},\frac{2z_0}{c^2}\right) = \frac{2x_0}{a^2}+\frac{2y_0}{b^2}+\frac{2z_0}{c^2} = 0.
	\label{eq:32}
\end{equation}
由\cref{eq:32}和原始方程三个方程联立求解得
\begin{equation}
	\frac{z^2 \,{\left(a^6 \,b^2 +a^6 \,c^2 -2\,a^4 \,b^4 -2\,a^4 \,c^4 +a^2 \,b^6 +a^2 \,c^6 +b^6 \,c^2 -2\,b^4 \,c^4 +b^2 \,c^6 \right)}}{a^2 \,b^2 \,c^2 \,{{\left(a^2 -b^2 \right)}}^2 }=0.
\end{equation}
要和$z$无关,则
\begin{equation}
	a^6 \,b^2 +a^6 \,c^2 -2\,a^4 \,b^4 -2\,a^4 \,c^4 +a^2 \,b^6 +a^2 \,c^6 +b^6 \,c^2 -2\,b^4 \,c^4 +b^2 \,c^6=0,
\end{equation}
即
\begin{equation}
	\left(a^3b-b^3a\right)^2+\left(b^3c-c^3b\right)^2+\left(c^3a-a^3c\right)^2=0,
\end{equation}
所以
\begin{equation}
	a^2=b^2=c^2.
\end{equation}
代入原方程,满足要求。




\section{}

\subsection{1}

设 $f \in C^{2}\left(\mathbb{R}^{n}, \mathbb{R}^{n}\right)$, 且在 $\mathbb{R}^{n}$ 上 $\operatorname{det} \Dif f(x) \neq 0$, 又当 $|x| \rightarrow+\infty$ 时, $|f(x)| \rightarrow+\infty$.

求证: $f\left(\mathbb{R}^{n}\right)=\mathbb{R}^{n}$ .


\begin{proof}

	即证 $\forall \xi \in \mathbb{R}^{n},\ \exists x_{0} \in \mathbb{R}^{n},$ s.t. $f\left(x_{0}\right)=\xi$

	$\because|x| \rightarrow+\infty$ 时, $|f(x)| \rightarrow+\infty$,

	$\therefore \forall \xi \in \mathbb{R}^{n},\ |x| \rightarrow+\infty$ 时, $|f(x)-\xi| \rightarrow+\infty$,

	$\therefore \exists x_{0} \in \mathbb{R}^{n}, \ \mathrm{s.t.}\ |f(x)-\xi|$ 取最小值,即 $|f(x)-\xi|^{2}$ 取最小值.

	$\therefore \frac{\partial}{\partial x_{i}}|f(x)-\xi|^{2}=0,$ 当 $x=x_{0}$,

	即有
$\Dif f\left(x_{0}\right) \cdot\left(f\left(x_{0}\right)-\xi\right)=0$.

又 $\operatorname{det} \Dif f(x) \neq 0,\  \therefore f\left(x_{0}\right)-\xi=0,$ 即 $f\left(x_{0}\right)=\xi$.
\end{proof}


\subsection{2}

设 ${D}$ 为有界凸域, 二元函数 $f(x, y)$ 在 $\bar{D}$ 上连续, 在边界上为常数, 在 ${D}$ 内可微.

求证:
${D}$ 内一定有一函数的临界点.

\begin{proof}
$f(x, y)$ 在 $\bar{D}$ 上连续 $\Rightarrow$

$f(x, y)$ 在 $\bar{D}$ 上可取到最大值和最小 $f_{\max }=f\left(x_{1}, y_{1}\right), f_{\min }=f\left(x_{2}, y_{2}\right)$ .

若 $\left(x_{1}, y_{1}\right),\left(x_{2}, y_{2}\right)$ 同在边界上, 则 $f_{\max }=f_{\min }, f=C, \quad f_{x} \equiv 0, f_{y} \equiv 0$ .

若 $\left(x_{1}, y_{1}\right),\left(x_{2}, y_{2}\right)$ 不同在边界上, 则 ${D}$ 内一定有一极值点 $\left(x_{0}, y_{0}\right),$ 则
\begin{equation}
	f_{x}\left(x_{0}, y_{0}\right)=0,\quad f_{y}\left(x_{0}, y_{0}\right)=0.
\end{equation}
\end{proof}

\subsection{3}
\begin{enumerate}[(1)]
	\item 求 $f(x, y, z)=x^{a} y^{b} z^{c}$ 在约束条件 $x+y+z=1$ 下的最大值, 其中 ${a}, {b}, {c}$ 是正常数,
	$x,y,z$ 非负.
	\item 证明对六个正数 $a,b,c,u,v,w$
	\begin{equation}
		\left(\frac{u}{a}\right)^{a}\left(\frac{v}{b}\right)^{b}\left(\frac{w}{c}\right)^{c} \leq\left(\frac{u+v+w}{a+b+c}\right)^{a+b+c}
	\end{equation}
	成立.
\end{enumerate}



\begin{enumerate}[(1)]
	\item {\sf \bfseries{解:}} $ f(x, y, z)=x^{a} y^{b} z^{c}=x^{a} y^{b}(1-x-y)^{c}$

	$\ln f=a \ln x+b \ln y+c \ln (1-x-y)$
	
	$f$ 的最大值即 $\ln f$ 的最大值
	
	求导
	\begin{equation}
		\left\{\begin{array}{l}\frac{a}{x}-\frac{c}{1-x-y}=0 \\ \frac{b}{y}-\frac{c}{1-x-y}=0\end{array} \Rightarrow \frac{a}{x}=\frac{b}{y}=\frac{c}{1-x-y}\right.
	\end{equation}
	可得 $x=\frac{a}{a+b+c}, y=\frac{b}{a+b+c}, z=\frac{c}{a+b+c}$ 时取最大值 $\frac{a^{a} b^{b} c^{c}}{(a+b+c)^{a+b+c}}$ .
	\item \begin{proof}
		令 $x=\frac{u}{u+v+w},\ y=\frac{v}{u+v+w},\ z=\frac{w}{u+v+w}$ 则由(1) 有
		\begin{equation}
			\left(\frac{u}{u+v+w}\right)^{a}\left(\frac{v}{u+v+w}\right)^{b}\left(\frac{w}{u+v+w}\right)^{c} \leqslant \frac{a^{a} b^{b} c^{c}}{(a+b+c)^{a+b+c}},
		\end{equation}
		即
		\begin{equation}
			\left(\frac{u}{a}\right)^{a}\left(\frac{v}{b}\right)^{b}\left(\frac{w}{c}\right)^{c} \leqslant \left(\frac{u+v+w}{a+b+c}\right)^{a+b+c}.
		\end{equation}
	\end{proof}
\end{enumerate}

\subsection{4}

设 $u(x, y)$ 在 $x^{2}+y^{2} \leq 1$ 上连续, 在 $x^{2}+y^{2}<1$ 上满足:
\begin{equation}
	\frac{\partial^{2} u}{\partial x^{2}}+\frac{\partial^{2} u}{\partial y^{2}}=u,
\end{equation}
且在 $x^{2}+y^{2}=1$ 上 $u(x, y)>0,$ 证明 :
\begin{enumerate}[(1)]
	\item 当 $x^{2}+y^{2} \leq 1$ 时, $u(x, y) \geq 0$;
	\item 当 $x^{2}+y^{2} \leq 1$ 时, $u(x, y)>0$.
\end{enumerate}

\begin{proof}
	~
	\begin{enumerate}[(1)]
		\item 上可取得最小值 $u_{0}=u\left(x_{0}, y_{0}\right)$ .
		
		反证: 设 $\exists\left(x^{\prime}, y^{\prime}\right),$ 使得 $u\left(x^{\prime}, y^{\prime}\right)<0,$ 则 $u_{0}=u\left(x_{0}, y_{0}\right)<0,$ 且 $x_{0}^{2}+y_{0}^{2}<1$ .

		所以
		\begin{gather}
			\frac{\partial u\left(x_{0}, y_{0}\right)}{\partial x}=\frac{\partial u\left(x_{0}, y_{0}\right)}{\partial y}=0,\\
			\frac{\partial^{2} u\left(x_{0}, y_{0}\right)}{\partial x^{2}} \geqslant 0, \frac{\partial^{2} u\left(x_{0}, y_{0}\right)}{\partial y^{2}} \geqslant 0,
		\end{gather}
		即 ${H}_{u}\left(x_{0}, y_{0}\right)$ 至少是半正定.

		$\frac{\partial^{2} u\left(x_{0}, y_{0}\right)}{\partial x^{2}}+\frac{\partial^{2} u\left(x_{0}, y_{0}\right)}{\partial y^{2}}=u\left(x_{0}, y_{0}\right) \geqslant 0,$ 与 $u_{0}=u\left(x_{0}, y_{0}\right)<0$ 矛盾, 所以
		$u(x, y) \geqslant 0$.
		\item 已知 $: \frac{\partial^{2} C \me^{x}}{\partial x^{2}}+\frac{\partial^{2} C \me^{x}}{\partial y^{2}}=C \me^{x},$ 可得
		\begin{equation}
			\frac{\partial^{2} u-C \me^{x}}{\partial x^{2}}+\frac{\partial^{2} u-C \me^{x}}{\partial y^{2}}=u-C \me^{x}.
		\end{equation}
		
		令 $v=u-C \me^{x},$ 则 $\frac{\partial^{2} v}{\partial x^{2}}+\frac{\partial^{2} v}{\partial y^{2}}=v$,
		
		因为在 $x^{2}+y^{2}=1$ 上 $u(x, y)>0,$ 所以在 $x^{2}+y^{2}=1$ 存在最小值 $\bar{u}>0$.

		取 $C$ 使得 $C>0,$ 且 $\bar{u}-C \me>0,$ 则在 $x^{2}+y^{2}=1$ 上,  $v>0$ .

		由(1)可知: 当 $x^{2}+y^{2} \leq 1$ 时,  $v(x, y) \geq 0,$ 而 $u=v+C \me^{x}>0$ .
	\end{enumerate}
\end{proof}





\section{}
\subsection{1}

若 $F(x, y, z)=0$ 可分别解出 $x=f(y, z),\ y=g(z, x),\ z=h(x, y)$, 则 $f_{z} g_{x} h_{y}=-1$.

\begin{proof}
	$\quad F(x, y, z)=F(f(y, z), y, z)=0$
	对 ${z}$ 求偏导
	$F_{x} \cdot f_{z}+F_{z}=0$,

	$\therefore f_{z}=-\frac{F_{z}}{F_{x}}$,

	同理 $g_{x}=-\frac{F_{x}}{F_{y}},\ h_{y}=-\frac{F_{y}}{F_{z}}$.

	$\therefore f_{z} \cdot g_{x} \cdot h_{y}=-1$.
\end{proof}

\subsection{2}


讨论二元函数
\begin{equation}
	f(x, y)=\left\{\begin{aligned}&\frac{|x|^{\alpha}|y|^{\beta}}{x^{2}+y^{2}}\left(x^{2}+y^{2} \neq 0\right), \\& 0\qquad \quad  \left(x^{2}+y^{2}=0\right).\end{aligned}\right.
\end{equation}
的连续性,可导性与可微性。

{ \sf \bfseries 解:} 令 $g(x, y)=\frac{|x|^{\alpha}|y|^{\beta}}{x^{2}+y^{2}}$.

\subsubsection{连续性}


$\alpha<0$ 或 $\beta<0$ 时, 不连续.

$\alpha \geq 0,\ \beta \geq 0$ 时, 令 $x=r \cos \theta,\ y=r \sin \theta$, 则
\begin{equation}
	g(x, y)=\frac{|x|^{\alpha}|y|^{\beta}}{x^{2}+y^{2}}=|r|^{\alpha+\beta-2}|\cos \theta|^{\alpha}|\sin \theta|^{\beta},
\end{equation}

$\alpha+\beta>2$ 时, $\lim \limits_{r \rightarrow 0} g=0$, 连续,

$\alpha+\beta=2$ 时, $\lim \limits_{r \rightarrow 0} g=|\cos \theta|^{\alpha}|\sin \theta|^{\beta}$, 不连续,

$\alpha+\beta<2$ 时,不连续,

$\therefore \alpha \geq 0, \beta \geq 0$, 且 $\alpha+\beta>2$ 时连续.

\subsubsection{可导性}


考察
\begin{equation}
	\lim \limits_{r \rightarrow 0} \frac{f(r \cos \theta, r \sin \theta)-f(0,0)}{r}=\lim \limits_{r \rightarrow 0} \frac{g}{r}=\lim \limits_{r \rightarrow 0} \frac{|r|^{\alpha+\beta-2}|\cos \theta|^{\alpha}|\sin \theta|^{\beta}}{r}.
\end{equation}
$\alpha+\beta>3$ 时, $\lim \limits_{r \rightarrow 0} \frac{g}{r}=0$, 可导,

$\alpha+\beta=3$ 时, $\lim \limits_{r \rightarrow 0+} \frac{g}{r}=|\cos \theta|^{\alpha}|\sin \theta|^{\beta},\ \lim \limits_{r \rightarrow 0-} \frac{g}{r}=-|\cos \theta|^{\alpha}|\sin \theta|^{\beta}$, 不可导,

$\alpha+\beta<3$ 时,不可导,

$\therefore \alpha \geq 0, \beta \geq 0$, 且 $\alpha+\beta>3$ 时可导.

\subsubsection{可微性}

若可微, 则各方向导数存在, 由可导性 $\alpha \geq 0,\ \beta \geq 0,\ \alpha+\beta>3$, 且 $\frac{\partial f}{\partial x}=0,\ \frac{\partial f}{\partial y}=0$,

$\therefore f(x, y)=o\left(\sqrt{x^{2}+y^{2}}\right) \Rightarrow \alpha+\beta>3$,

$\therefore \alpha \geq 0,\ \beta \geq 0$, 且 $\alpha+\beta>3$ 时可微.

\subsection{3}

求函数 $u=x^{3}+y^{3}+z^{3}-2 x y z$ 在单位球内部 $x^{2}+y^{2}+z^{2} \leq 1$ 的最大值与最小值。


{ \sf \bfseries 解:} 最大值与最小值在极值点或边界上取得.
\begin{equation}
	\left\{\begin{array}{l}\frac{\partial u}{\partial x}=0 \\ \frac{\partial u}{\partial y}=0, \Rightarrow x=y=z=0,\\ \frac{\partial u}{\partial z}=0.\end{array}\right.
\end{equation}
对任意 $x=y=z=\varepsilon>0, u>0$,对任意 $x=y=z=-\varepsilon<0, u<0$, 所以临界点 $(0,0,0)$
不是极值点,最大最小值在边界上取得

设 $f=x^{3}+y^{3}+z^{3}-2 x y z+\lambda\left(x^{2}+y^{2}+z^{2}-1\right)$,
\begin{equation}
	\left\{\begin{array}{l}\frac{\partial f}{\partial x}=3 x^{2}-2 y z+2 x \lambda=0, \\ \frac{\partial f}{\partial y}=3 y^{2}-2 x z+2 y \lambda=0, \\ \frac{\partial f}{\partial z}=3 z^{2}-2 x y+2 z \lambda=0, \\ \frac{\partial f}{\partial \lambda}=x^{2}+y^{2}+z^{2}-1=0.\end{array}\right.
\end{equation}
$\Rightarrow(x, y, z)=\left(\pm \frac{\sqrt{3}}{3}, \pm \frac{\sqrt{3}}{3}, \pm \frac{\sqrt{3}}{3}\right) \quad$ 或 $\quad(0,0, \pm 1) \quad$ 或 $\quad\left(\frac{5 \sqrt{6}}{18}, \frac{5 \sqrt{6}}{18},-\frac{\sqrt{6}}{9}\right)$.

$\left(-\frac{5 \sqrt{6}}{18},-\frac{5 \sqrt{6}}{18}, \frac{\sqrt{6}}{9}\right) \text { 可轮换 }$.

	{ 在 } $(0,0,1),(0,1,0),(1,0,0)$  { 取最大值 } $1, $ { 在 } $(0,0,-1),(0,0,-1),(-1,0,0) $ { 取最小值 }$-1$.




\section{}


\subsection{1}

设一元函数 $f(x)$ 在 $[a, b]$ 上可积.在 $[a, b] \times[a, b]$ 上定义 $F(x, y)=[f(x)-f(y)]^{2}$
\begin{enumerate}[(1)]
	\item 将重积分 $\iint_{D} F(x, y) \dif x \dif y$ 化为累次积分;
	\item 证明: $\left[\int_{a}^{b} f(x) \dif x\right]^{2} \leqslant (b-a) \int_{a}^{b} f^{2}(x) \dif x$ .
\end{enumerate}

\begin{enumerate}[(1)]
	\item {\sf \bfseries{解:}}
	      \begin{equation}
		      \begin{aligned}
			      \iint_{D} F(x, y) \dif x \dif y & =\int_{a}^{b} \int_{a}^{b}\left[f^{2}(x)+f^{2}(y)-2 f(x) f(y)\right] \dif x \dif y                                                                     \\
			                                      & =\int_{a}^{b} \int_{a}^{b} f^{2}(x) \dif x \dif y+\int_{a}^{b} \int_{a}^{b} f^{2}(y) \dif x \dif y-2 \int_{a}^{b} \int_{a}^{b} f(x) f(y) \dif x \dif y \\
			                                      & =2(b-a) \int_{a}^{b} f^{2}(x) \dif x-2 \int_{a}^{b} f(x) \dif x \int_{a}^{b} f(y) \dif y                                                               \\
			                                      & =2(b-a) \int_{a}^{b} f^{2}(x) \dif x-2\left[\int_{a}^{b} f(x) \dif x\right]^{2}.
		      \end{aligned}
	      \end{equation}

	\item \begin{proof}
		      $F(x, y)=[f(x)-f(y)]^{2} \geqslant 0, \ \therefore \iint_{D} F(x, y) \dif x \dif y \geqslant 0$, 由(1)即有
		      \begin{equation}
			      \left[\int_{a}^{b} f(x) \dif x\right]^{2} \leqslant (b-a) \int_{a}^{b} f^{2}(x) \dif x.
		      \end{equation}
	      \end{proof}
\end{enumerate}

\subsection{2}

计算三重积分 $I=\int_{0}^{1} \dif x \int_{x}^{1}\dif y \int_{y}^{1} y \sqrt{1+z^{4}} \dif z$.

	{\sf \bfseries{解:}}
\begin{align}
	I & =\int_{0}^{1} \dif z \int_{0}^{z} \dif y \int_{0}^{y} y \sqrt{1+z^{4}} \dif x           \\
	  & =\int_{0}^{1} \dif z \int_{0}^{z} y^{2} \sqrt{1+z^{4}} \dif y                           \\
	  & =\frac{1}{3} \int_{0}^{1} z^{3} \sqrt{1+z^{4}} \dif z                                   \\
	  & =\frac{1}{12} \int_{0}^{1} \sqrt{1+z^{4}} \dif \left(1+z^{4}\right)                     \\
	  & =\left.\frac{1}{12} \cdot \frac{2}{3}\left(1+z^{4}\right)^{\frac{3}{2}}\right|_{0} ^{1} \\
	  & =\frac{2 \sqrt{2}-1}{18}.
\end{align}



\subsection{3}

化重积分为累次计分 $\iiint_{V} f \dif V$ , 其中 ${V}$ 是 $x^{2}+y^{2}=1, \ y^{2}+z^{2}=1, \ x^{2}+z^{2}=1$ 围成的区域.

{\sf \bfseries{解:}}
\begin{equation}
	\begin{aligned}
		\iiint_{V} f \dif V= & \int_{-\frac{\sqrt{2}}{2}}^{\frac{\sqrt{2}}{2}} \dif y \int_{|y|}^{\sqrt{1-y^{2}}} \dif x \int_{-\sqrt{1-x^{2}}}^{\sqrt{1-x^{2}}} f \dif z+\int_{-\frac{\sqrt{2}}{2}}^{\frac{\sqrt{2}}{2}} \dif y \int_{-\sqrt{1-y^{2}}}^{-|y|} \dif x \int_{-\sqrt{1-x^{2}}}^{\sqrt{1-x^{2}}} f \dif z   \\
		                     & +\int_{-\frac{\sqrt{2}}{2}}^{\frac{\sqrt{2}}{2}} \dif x \int_{|x|}^{\sqrt{1-x^{2}}} \dif y \int_{-\sqrt{1-y^{2}}}^{\sqrt{1-y^{2}}} f \dif z+\int_{-\frac{\sqrt{2}}{2}}^{\frac{\sqrt{2}}{2}} \dif x \int_{-\sqrt{1-x^{2}}}^{-|x|} \dif y \int_{-\sqrt{1-y^{2}}}^{\sqrt{1-y^{2}}} f \dif z.
	\end{aligned}
\end{equation}


\section{}


\subsection{1}

设在 $D=[a, b] \times[c, d]$ 上定义的二元函数 $f(x, y)$ 有二阶连续偏导数
\begin{enumerate}[(1)]
	\item 证明: $\iint_{D} f_{x y}^{\prime \prime}(x, y) \dif x \dif y=\iint_{D} f_{y x}^{\prime \prime}(x, y) \dif x \dif y,\ \forall(x, y) \in D$;
	\item 利用 (1) 证明: $f_{x y}^{\prime \prime}(x, y)=f_{y x}^{\prime \prime}(x, y),\ \forall(x, y) \in D$,
\end{enumerate}



\begin{proof}
	~
	\begin{enumerate}[(1)]
		\item
		      \begin{equation}
			      \begin{aligned}
				      \iint_{D} f_{x y}^{\prime \prime}(x, y )\dif x \dif y= & \ \int_{a}^{b} \dif x \int_{c}^{d}f_{x y}^{\prime \prime}(x, y) \dif y        \\
				      =                                                      & \ \int_{a}^{b}   \left(f_{x}^{\prime}(x, d)-f_{x}^{\prime}(x, c)\right)\dif x \\
				      =                                                      & f(b, d)-f(b, c)-f(a, d)+f(a, c).
			      \end{aligned}
		      \end{equation}
		      同理
		      \begin{equation}
			      \iint_{D} f_{y x}^{\prime \prime} \dif x \dif y=f(b, d)-f(b, c)-f(a, d)+f(a, c)
		      \end{equation}
		      所以
		      \begin{equation}
			      \iint_{D} f_{y x}^{\prime \prime}(x, y) \dif x \dif y=\iint_{D} f_{x y}^{\prime \prime}(x, y) \dif x \dif y.
		      \end{equation}

		\item 以上关系在任意 $D=[a, b] \times[c, d]$ 上成立.若 $F(x, y)$ 连续, 在任意
		      $D=[a, b] \times[c, d]$ 成立 $\iint_{D} F(x, y) \dif x \dif y=0, $ 则有 $F(x, y) \equiv 0$.

		      假设 $F(x, y) \equiv 0$ 不成立, 即在某点 $\left(x_{0}, y_{0}\right), F\left(x_{0}, y_{0}\right) \neq 0,$ 不妨设 $F\left(x_{0}, y_{0}\right)>0$.

		      由连续性,在 $\left(x_{0}, y_{0}\right)$ 某方形领域 $D^{\prime}=\left[x_{0}-\delta, x_{0}+\delta\right] \times\left[y_{0}-\delta, y_{0}+\delta\right] $ 上  $F(x, y)>0$.

		      则 $\iint_{D^{\prime}} F(x, y) \dif x \dif y>0,$ 矛盾.所以 $F(x, y) \equiv 0$。

		      取 $F(x, y)=f_{y x}^{\prime \prime}(x, y)-f_{x y}^{\prime \prime}(x, y),$ 即可证明 $f_{y x}^{\prime \prime}(x, y)=f_{x y}^{\prime \prime}(x, y)$ .
	\end{enumerate}


\end{proof}


\subsection{2}

计算 $x^{2}+y^{2} \leq 1, \ z^{2}+y^{2} \leq 1, \ x^{2}+z^{2} \leq 1$ 围成区域的体积.

{\sf \bfseries{解:}}
\begin{equation}
	8(2-\sqrt{2}).
\end{equation}






\section{}

\subsection{1}

$f(x)\in C[-h,h],\ h=\sqrt{\alpha^2+\beta^2+\gamma^2}$,证明:
\begin{equation}
	\iint_S f(\alpha x+\beta y+\gamma z) \dif S = 2\pi \int^1_{-1}f(hu)\dif u,
\end{equation}
其中$S=x^2+y^2+z^2=1$.

\begin{proof}

	\begin{equation}
		\left\{
		\begin{aligned}
			\xi   & = a_1 x + b_1y + c_1z                    \\
			\eta  & = a_2 x + b_2y + c_2z                    \\
			\zeta & = \frac{1}{h}(\alpha x+\beta y+\gamma z)
		\end{aligned}
		\right.
	\end{equation}
	其中
	\begin{equation}
		\begin{pmatrix}
			a_1              & b_1             & c_1              \\
			a_2              & b_2             & c_2              \\
			\frac{\alpha}{h} & \frac{\beta}{h} & \frac{\gamma}{h} \\
		\end{pmatrix}
	\end{equation}
	是单位正交矩阵。令
	\begin{equation}
		\xi = \cos \theta \cos \varphi,\quad \eta = \sin \theta \sin \varphi,\quad \zeta = \sin \varphi.\qquad D:\begin{cases}
			0\leq \theta \leq 2\pi \\
			-\frac{\pi}{2} \leq \varphi \leq \frac{\pi}{2}
		\end{cases}
	\end{equation}
	\begin{equation}
		\begin{aligned}
			\iint_S f(\alpha x+\beta y+\gamma z) \dif S & = \iint_D f(h\zeta)\dif S                                                                                        \\
			                                            & = \iint_D f(h\zeta)\cos \varphi \dif \theta \dif \varphi                                                         \\
			                                            & = \int^{2\pi}_{0} \dif \theta  \int^{\frac{\pi}{2}}_{-\frac{\pi}{2}} f(h \sin \varphi)\cos \varphi  \dif \varphi \\
			                                            & = 2\pi \int^1_{-1}f(hu)\dif u,
		\end{aligned}
	\end{equation}



\end{proof}

\subsection{2}

在无穷大三维空间中,半径为$R$的球面上均匀分布着电荷密度为$\rho$的电荷,求任一空间点的电势。


\begin{equation}
	\left\{
	\begin{aligned}
		x & = R\cos \theta \cos \varphi  \\
		y & = R \sin \theta \cos \varphi \\
		z & = R\sin \varphi
	\end{aligned}
	\right.
\end{equation}
\begin{equation}
	\begin{aligned}
		W(0,0,a) & = \iint_S \frac{\rho \dif S}{\sqrt{x^2+y^2+(z-a)^2}}                                                                                                                                  \\
		         & = \rho R^2 \int^{2\pi}_0 \dif \theta \int^{\frac{\pi}{2}}_{-\frac{\pi}{2}} \frac{\cos \varphi \dif \varphi}{\sqrt{R^2+a^2-2R\sin \varphi}}                                            \\
		         & = \rho R^2 \int^{2\pi}_0 \dif \theta \left(-\frac{1}{2Ra}\right) \int^{\frac{\pi}{2}}_{-\frac{\pi}{2}} \frac{\dif \left(R^2+a^2-2R\sin \varphi\right)}{\sqrt{R^2+a^2-2R\sin \varphi}} \\
		         & =\frac{2\pi\rho R}{a} \sqrt{R^2+a^2-2R\sin \varphi}\bigg|^{-\frac{\pi}{2}}_{\frac{\pi}{2}}                                                                                            \\
		         & =\frac{2\pi\rho R}{a}\left(R+a-\abs{R-a}\right)=\begin{cases}
			4\pi R\rho,              & 0<a<R   \\
			\frac{4\pi R^2}{a}\rho , & a\geq R
		\end{cases}
	\end{aligned}
\end{equation}

\subsection{3}

设 $f(x, y)$ 连续, $L$ 是一封闭的分段光滑简单曲线,设
\begin{equation}
	u(x, y)=\oint_{L} f(\xi, \eta) \ln \left(\frac{1}{\sqrt{(x-\xi)^{2}+(y-\eta)^{2}}}\right) \dif s,
\end{equation}
证明: $\lim \limits_{x \rightarrow \infty, y \rightarrow \infty} u(x, y)=0$ 的充要条件是 $\oint_{L} f(\xi, \eta) \dif s=0.$


\begin{proof}

	$f(x,y)$在$L$上有界,$\abs{f(x,y)}\leq k$,设$L$的长度为$S$。固定一点$(\xi_0,\eta_0)\in \alpha$.
	\begin{equation}
		\abs{\ln{\left(\frac{1}{\sqrt{(\xi-x)^2+(\eta-y)^2}}\right)}-\ln{\left(\frac{1}{\sqrt{(\xi_0-x)^2+(\eta_0-y)^2}}\right)}} = \abs{\ln{\left(\frac{\sqrt{(\xi_0-x)^2+(\eta_0-y)^2}}{\sqrt{(\xi-x)^2+(\eta-y)^2}}\right)}}
	\end{equation}
	趋于0.

	\begin{equation}
		u(x,y) = \oint f(\xi, \eta) \ln \left(\frac{1}{\sqrt{(x-\xi_0)^{2}+(y-\eta_0)^{2}}}\right) + f(\xi, \eta)  \ln{\left(\frac{\sqrt{(\xi_0-x)^2+(\eta_0-y)^2}}{\sqrt{(\xi-x)^2+(\eta-y)^2}}\right)} \dif s
	\end{equation}
	所以
	\begin{equation}
		\abs{u(x,y)- \oint f(\xi, \eta) \ln \left(\frac{1}{\sqrt{(x-\xi_0)^{2}+(y-\eta_0)^{2}}}\right) \dif s} \leq kS\varepsilon
	\end{equation}
	又
	\begin{equation}
		\oint f(\xi, \eta) \ln \left(\frac{1}{\sqrt{(x-\xi_0)^{2}+(y-\eta_0)^{2}}}\right) \dif s =  \ln \left(\frac{1}{\sqrt{(x-\xi_0)^{2}+(y-\eta_0)^{2}}}\right)\oint f(\xi, \eta) \dif s
	\end{equation}
	所以$\lim \limits_{x \rightarrow \infty, y \rightarrow \infty} u(x, y)=0$ 的充要条件是 $\oint_{L} f(\xi, \eta) \dif s=0.$

\end{proof}


\section{}
\subsection{1}

设 $S$ 为椭球面 $\frac{x^{2}}{a^{2}}+\frac{y^{2}}{b^{2}}+\frac{z^{2}}{c^{2}}=1,\ \vec{r}=x \vec{i}+y \vec{j}+z \vec{k},\ \vec{n}$ 为曲面 $S$ 的单位外法向, $d(x, y, z)$
表示原点到 $(x, y, z) \in S$ 处切平面的距离,求以下积分:
\begin{enumerate}
	\item $\oiint_{S} \vec{r} \cdot \vec{n} \dif \sigma$
	\item $\oiint_{S} d(x, y, z) \dif \sigma$
	\item $\oiint_{S} \frac{\dif \sigma}{d(x, y, z)}$
\end{enumerate}


\subsubsection{1}

\begin{equation}
	\vec{n}=\frac{1}{\sqrt{\frac{x^{2}}{a^{2}}+\frac{y^{2}}{b^{4}}+\frac{z^{2}}{c^{4}}}}\left(\frac{x}{a^{2}}, \frac{y}{b^{2}}, \frac{z}{c^{2}}\right).
\end{equation}
\begin{equation}
	\oiint_S \vec{n} \cdot \vec{r} \dif \sigma =\iiint \nabla \cdot \vec{r} \dif V=3 V=4 \pi a b c.
\end{equation}


\subsubsection{2}

$\because d = \vec{n} \cdot \vec{r},\ \therefore$
\begin{equation}
	\oiint_{S} d(x, y, z) \dif \sigma=4 \pi a b c.
\end{equation}


\subsubsection{3}

\begin{equation}
	\begin{aligned}
		\frac{1}{d} & =\sqrt{\frac{x^{2}}{a^{4}}+\frac{y^{2}}{b^{4}}+\frac{z^{2}}{c^{4}}}=\sqrt{\frac{1}{a^{2}} \cos ^{2} \theta \sin ^{2} \varphi+\frac{1}{b^{2}} \sin ^{2} \theta \sin ^{2} \varphi+\frac{1}{c^{2}} \cos ^{2} \varphi} \\
		            & =\frac{1}{a b c} \sqrt{b^{2} c^{2} \cos ^{2} \theta \sin ^{2}\varphi+a^{2} c \sin ^{2} \theta \sin ^{2} \varphi+a^{2} {b}^{2} \cos ^{2} \varphi}.
	\end{aligned}
\end{equation}
\begin{equation}
	\dif \sigma=\abs{\bm{r}_{\theta}\times\bm{r}_{\varphi}}\dif \theta \dif \varphi=\sqrt{b^{2}c^{2} \cos^{2} \theta \sin ^{4}\varphi + a^{2} c^{2} \sin ^{2} \theta \sin ^{4} \varphi + a^2b^2\sin ^2\varphi \cos ^2 \varphi } \dif \theta \dif \varphi
\end{equation}
\begin{equation}
	\begin{aligned}
		  & \oiint_{S} \frac{\dif \sigma}{d(x, y, z)}                           \\  =&\ \frac{4}{abc}\int^{\frac{\pi}{2}}_0\dif \theta \int^{\pi}_0|\sin \varphi| \left(b^{2}c^{2} \cos^{2} \theta \sin ^{2}\varphi + a^{2} c^{2} \sin ^{2} \theta \sin ^{2} \varphi + a^2b^2 \cos ^2 \varphi \right) \dif \varphi\\
		= & \ \frac{4\pi}{3abc}\left(a^{2} b^{2}+b^{2} c^{2}+c^{2}a^{2} \right)
	\end{aligned}
\end{equation}


\subsection{2}

考虑空间 $\mathbb{R}^{3}$ 中在原点电量为 ${Q}$ 的电荷在 $\vec{r}=(x, y, z)$ 处产生的电场: $\vec{E}=\frac{1}{4\pi \varepsilon_0} \frac{{Q} \vec{r}}{r^{3}}$, 这里
$r=\sqrt{x^{2}+y^{2}+z^{2}}$, 设 $\Omega$ 是 ${R}^{3}$ 的开区域, $\partial \Omega$ 充分光滑, $\vec{n}$ 为 $\Omega$ 的外法向,证明:
\begin{equation}
	\oiint_{\partial \Omega} \vec{E} \cdot \vec{n} \dif \sigma=\left\{\begin{array}{ll}0, & (0,0,0) \notin \Omega \\  \frac{Q}{\varepsilon_0}, & (0,0,0) \in \Omega\end{array}\right.
\end{equation}

\begin{proof}
	若电荷在$\Omega$内。

	取一个以$\vec{r}=(x, y, z)$为球心的半径为$a$的小球面$S$,使得$S$在$\partial \Omega$内,对于两曲面中间的区域$V$有
	\begin{equation}
		\oiint_{\partial \Omega} \vec{E} \cdot \vec{n} \dif \sigma-\oiint_{S} \vec{E} \cdot \vec{n} \dif \sigma=\iiint_V \nabla \cdot \vec{E} \dif v =\iiint_V \nabla \cdot \frac{1}{4\pi \varepsilon_0} \frac{{Q} \vec{r}}{r^{3}} \dif v = 0
	\end{equation}
	\begin{equation}
		\oiint_{\partial \Omega} \vec{E} \cdot \vec{n} \dif \sigma  =\oiint_{S} \vec{E} \cdot \vec{n} \dif \sigma    =\frac{1}{4\pi \varepsilon_0} Q \oiint_S \frac{\vec{r}}{a^3}\cdot \vec{n} \dif \sigma  = \frac{Q}{\varepsilon_0}
	\end{equation}

	若电荷不在$\Omega$内,则$\Omega$内无暇点。
	\begin{equation}
		\oiint_{\partial \Omega} \vec{E} \cdot \vec{n} \dif \sigma=\iiint_{\Omega} \nabla \cdot \frac{1}{4\pi \varepsilon_0} \frac{{Q} \vec{r}}{r^{3}} \dif v = 0
	\end{equation}

\end{proof}

\end{document}
