
\subsection{1}

求证:

$\lim \limits_{x \rightarrow+\infty,\ y \rightarrow-\infty}[f(x)+g(y)] \text { 存在的充要条件是 } \lim \limits_{x \rightarrow+\infty} f(x) \text { 和 } \lim \limits_{y \rightarrow-\infty} g(y) \text { 同时存在. }$

\begin{proof}
	~
	\begin{enumerate}[(1)]
		\item 充分条件
		
		$\text { 设 } \lim \limits_{x \rightarrow+\infty,\ y \rightarrow-\infty}[f(x)+g(y)]=A$,

		$\forall \varepsilon>0,\ \exists N_{1}, N_{2}>0,$ 当 $x_{1}>N_{1},\ y_{i}<-N_{2},(i=1,2)$ 时有

		\begin{equation}
			\left|f\left(x_{1}\right)+g\left(y_{i}\right)-A\right|<\frac{\varepsilon}{2},
		\end{equation}
	$\therefore \forall \varepsilon>0,\ \exists N_{2}>0,\ \forall y_{1}, y_{2}<-N_{2} \text { 有 }$
	\begin{align}
		\abs{g\left(y_{1}\right)-g\left(y_{2}\right)}  &=\left|\left(f\left(x_{1}\right)+g\left(y_{1}\right)-A\right)-\left(f\left(x_{1}\right)+g\left(y_{2}\right)-A\right)\right| \\
		&<\left|f\left(x_{1}\right)+g\left(y_{1}\right)-A\right|+\left|f\left(x_{1}\right)+g\left(y_{2}\right)-A\right|<\varepsilon 
	\end{align}
	$\therefore \lim \limits_{y \rightarrow-\infty} g(y) \text { 存在 }$
	
	同理可证 $\lim \limits_{x \rightarrow+\infty} f(x)$ 存在.

	\item 必要条件
	
	若 $\lim \limits_{x \rightarrow+\infty} f(x)$ 和 $\lim \limits_{y \rightarrow-\infty} g(y)$ 同时存在,设 $\lim \limits_{x \rightarrow+\infty} f(x)=a, \quad \lim \limits_{y \rightarrow-\infty} g(y)=b,$ 则

	$\forall \varepsilon>0,\ \exists N_{1}>0,$ 当 $x>N_{1}$ 时 $|f(x)-a|<\frac{\varepsilon}{2},\ \exists N_{2}>0,$ 当 $y<-N_{2}$ 时, $|g(y)-b|<\frac{\varepsilon}{2}$

	则 $|f(x)+g(y)-(a+b)| \leqslant |f(x)-a|+|g(y)-b|<\varepsilon,$ 即有
	\begin{equation}
		\lim _{x \rightarrow+\infty,\ y \rightarrow-\infty}[f(x)+g(y)]=a+b.
	\end{equation}

	\end{enumerate}

\end{proof}

\subsection{2}


设二元函数 $f(x, y)$ 在圆周 $C:\left(x-x_{0}\right)^{2}+\left(y-y_{0}\right)^{2}=R^{2}$ 上连续.

证明: $f(x, y)$ 在 $C$
上达到上确界 ${M}$ 和下确界 ${m},$ 且取属于 $(m, M)$ 的值至少两次.

\begin{proof}
	~
	\begin{enumerate}[(1)]
		\item 在圆周 $C:\left(x-x_{0}\right)^{2}+\left(y-y_{0}\right)^{2}=R^{2}$ 上
		\begin{equation}
			\left\{\begin{array}{l}x=x(t)=x_{0}+R \cos t \\ y=y(t)=y_{0}+R \sin t\end{array},\ t \in[\theta, 2 \pi+\theta]\right.
		\end{equation}
		
		则 $f(x, y)=f(x(t), y(t))=g(t)$ 连续.

		$g(t)$ 连续性证明: $x, y$ 是 ${t}$ 的连续函数, $\therefore \forall \delta>0,\ \exists \delta_{1}, \delta_{2}>0,$ 当 $\left|t-t^{*}\right|<\delta_{1}$ 时有
		$\left|x-x^{*}\right|<\frac{\delta}{2},$ 当 $\left|t-t^{*}\right|<\delta_{2}$ 时有 $\left|y-y^{*}\right|<\frac{\delta}{2}$ .又由 $f(x, y)$ 连续性, $\forall \varepsilon>0,\ \exists \delta>0,$ 只
		要 $(x, y) \in C,\ \sqrt{\left(x-x^{*}\right)^{2}+\left(y-y^{*}\right)^{2}}<\delta,$ 就有 $\left|f(x, y)-f\left(x^{*}, y^{*}\right)\right|<\varepsilon$ .
		
		因此 $\forall \varepsilon>0,\ \exists \delta_{0}=\min \left\{\delta_{1}, \delta_{2}\right\},$ 当 $\left|t-t^{*}\right|<\delta_{0}$ 时有
		\begin{equation}
			(x, y) \in C, \sqrt{\left(x-x^{*}\right)^{2}+\left(y-y^{*}\right)^{2}}<\sqrt{\left(\frac{\delta}{2}\right)^{2}+\left(\frac{\delta}{2}\right)^{2}}<\delta,
		\end{equation}
		
		进而有 $\left|g(t)-g\left(t^{*}\right)\right|=\left|f(x, y)-f\left(x^{*}, y^{*}\right)\right|<\varepsilon, \quad \therefore g(t)$ 是 ${t}$ 的连续函数. 

		$\therefore g(t)$ 在闭区间 $[\theta, 2 \pi+\theta]$ 有界,且 $\exists t_{1},\ t_{2} \in[\theta, 2 \pi+\theta],\ \mathrm{s.t.} g\left(t_{1}\right)=m,\ g\left(t_{2}\right)=M$( 最大值与最小值定理).即 $f\left(x_{1}, y_{1}\right)=m,\  f\left(x_{2}, y_{2}\right)=M$, 其中 $x_{i}=x\left(t_{i}\right),\ y_{i}=y\left(t_{i}\right),\ (i=1,2)$.
		\item 若 $m=M$,则结论显然.
		$m \neq M$ 时,设 $g\left(\theta_{1}\right)=g\left(\theta_{1}+2 \pi\right)=f\left(x_{1}, y_{1}\right)=m,\ g\left(\theta_{2}\right)=f\left(x_{2}, y_{2}\right)=M,$ 则圆周可
		分为两段
		\begin{equation}
			\Gamma_{1}:\left\{\begin{array}{l}x=x(t)=x_{0}+R \cos t \\ y=y(t)=y_{0}+R \sin t\end{array}, t \in\left[\theta_{1}, \theta_{2}\right]\right. ; 
		\end{equation}
		\begin{equation}
			\Gamma_{2}:\left\{\begin{array}{l}x=x(t)=x_{0}+R \cos t \\ y=y(t)=y_{0}+R \sin t\end{array}, t \in\left[\theta_{2}, \theta_{1}+2 \pi\right]\right..
		\end{equation}
		
		则 $\forall \mu \in(m, M),\ \exists t_{1}^{*} \in\left(\theta_{1}, \theta_{2}\right), t_{2}^{*} \in\left(\theta_{2}, \theta_{1}+2 \pi\right),$ 满足 $g\left(t_{1}^{*}\right)=g\left(t_{2}^{*}\right)=\mu$.相应即有 $f\left(x_{1}^{*}, y_{1}^{*}\right)=f\left(x_{2}^{*}, y_{2}^{*}\right)=\mu$.
	\end{enumerate}
\end{proof}




\subsection{3}

证明: 若 $f(x, y)$ 分别对每一个变量 ${x}, {y}$ 是连续的,且对其中一个单调,则 $f(x, y)$ 是二元连续函数。

\begin{proof}
	不妨设 $f(x, y)$ 对 ${y}$ 单调。

$\forall \varepsilon>0,\ \exists \delta_{1}>0,\ \mathrm{s.t.}\  x_{0}<x<x_{0}+\delta_{1}$ 时
\begin{equation}
	\left|f\left(x, y_{0}\right)-f\left(x_{0}, y_{0}\right)\right|<\frac{\varepsilon}{4},
\end{equation}
$\forall \varepsilon>0,\ \exists \delta_{2}>0,\ \mathrm{s.t.}\ y_{0}<y<y_{0}+\delta_{2}$ 时
\begin{equation}
	\left|f\left(x_{0}, y_{0}\right)-f\left(x_{0}, y \right)\right|<\frac{\varepsilon}{4},
\end{equation}
$\forall \varepsilon>0,\ \exists \delta_{3}>0,\ \mathrm{s.t.}\ x_{0}<x<x_{0}+\delta_{3}$ 时
\begin{equation}
	\left|f\left(x, y_{0}+\delta_{2}\right)-f\left(x_{0}, y_{0}+\delta_{2}\right)\right|<\frac{\varepsilon}{4}.
\end{equation}

由以上三式可得 $\left|f\left(x, y_{0}+\delta_{2}\right)-f\left(x, y_{0}\right)\right|<\frac{3 \varepsilon}{4}$.

令 $\delta=\min \left\{\delta_{1}, \delta_{2}, \delta_{3}\right\},$ 当 $x \in\left(x_{0}, x_{0}+\delta\right), y \in\left(y_{0}, y_{0}+\delta\right)$ 时,由 ${f}$ 关于 ${y}$ 的单调性,
\begin{equation}
	\left|f(x, y)-f\left(x, y_{0}\right)\right|<\left|f\left(x, y_{0}+\delta_{2}\right)-f\left(x, y_{0}\right)\right|<\frac{3 \varepsilon}{4}.
\end{equation}
$\therefore\left|f(x, y)-f\left(x_{0}, y_{0}\right)\right|<\varepsilon$.

对于 $\left(x_{0}, y_{0}\right)$ 左方和下方的邻域内类似有同上结论。 $\therefore f(x, y)$ 二元连续。


\end{proof}
