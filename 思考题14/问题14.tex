\subsection{作业}
21.3.1(1)(2)(3), 21.3.3(1)(2)(3)(4), 21.3.4(1)(2), 21.3.5(2)(4), 22.5.1(1)(2)(3)(4)


\subsection{1}

设 $S$ 为椭球面 $\frac{x^{2}}{a^{2}}+\frac{y^{2}}{b^{2}}+\frac{z^{2}}{c^{2}}=1,\ \vec{r}=x \vec{i}+y \vec{j}+z \vec{k},\ \vec{n}$ 为曲面 $S$ 的单位外法向, $d(x, y, z)$
表示原点到 $(x, y, z) \in S$ 处切平面的距离,求以下积分:
\begin{enumerate}
	\item $\oiint_{S} \vec{r} \cdot \vec{n} \dif \sigma$
	\item $\oiint_{S} d(x, y, z) \dif \sigma$
	\item $\oiint_{S} \frac{\dif \sigma}{d(x, y, z)}$
\end{enumerate}


\subsection{2}

考虑空间 $\mathbb{R}^{3}$ 中在原点电量为 $\mathbb{Q}$ 的电荷在 $\vec{r}=(x, y, z)$ 处产生的电场: $\vec{E}=\varepsilon \frac{\mathbb{Q} \vec{r}}{r^{3}}$, 这里
$r=\sqrt{x^{2}+y^{2}+z^{2}}$, 设 $\Omega$ 是 $\mathbb{R}^{3}$ 的开区域, $\partial \Omega$ 充分光滑, $\vec{n}$ 为 $\Omega$ 的外法向,证明:
\begin{equation}
	\oiint_{\partial \Omega} \vec{E} \cdot \vec{n} \dif \sigma=\left\{\begin{array}{ll}0 & (0,0,0) \notin \Omega \\ 4 \pi \varepsilon \mathbb{Q} & (0,0,0) \in \Omega\end{array}\right.
\end{equation}
